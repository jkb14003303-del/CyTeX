\documentclass[dvipdfmx,report,uplatex]{jsbook}

% ===== 基本レイアウト =====
\usepackage[a4paper,top=30mm,bottom=30mm,left=30mm,right=30mm]{geometry}
% ===== フォント(和文HaranoAji)=====
\usepackage[haranoaji]{pxchfon}

% ===== 数式・図表 =====
\usepackage[version=4]{mhchem}
\usepackage{amsmath,amssymb,amsfonts,bm}
\usepackage[dvipdfmx]{graphicx}
\usepackage{placeins}
\usepackage[dvipdfmx]{graphicx}

% \graphicspath{{figures/}}
\usepackage[format=hang]{subcaption}
\usepackage{caption}
\captionsetup{labelsep=space}

% ===== 文献 =====
\usepackage[numbers,square]{natbib}

\setcitestyle{super}

% 単位(siunitx v3)
\usepackage{siunitx}
\sisetup{
  per-mode = power,
  propagate-math-font = true,
  reset-math-version = false,
  reset-text-family = false,
  reset-text-series = false,
  reset-text-shape = false,
  text-family-to-math = true,
  text-series-to-math = true,
  exponent-product = \times,
  output-decimal-marker = {.}
}
\sisetup{
  range-phrase = {\,\text{~}\,},
  range-units  = single
}

% ===== しおり・ハイパーリンク =====
\usepackage[
  dvipdfmx,
  setpagesize=false,
  bookmarks=true,
  bookmarksnumbered=true,
  bookmarksdepth=3,
  colorlinks=true,
  linkcolor = blue,
  pdfencoding=unicode,
  pdftitle={タイトル},
  pdfauthor={著者名},
  pdfsubject={サブジェクト},
  pdfkeywords={キーワード}
]{hyperref}
\usepackage{pxjahyper}


% ===== 見出し体裁 =====
\usepackage{titlesec}
\titleformat{\chapter}[hang]{\LARGE\gtfamily\bfseries}{第\arabic{chapter}章}{1em}{}
\titlespacing*{\chapter}{0pt}{0pt}{20pt}

% ===== 番号付け(章.節.連番)=====
\numberwithin{equation}{section}
\numberwithin{figure}{section}
\numberwithin{table}{section}
\renewcommand{\theequation}{\thechapter.\arabic{section}.\arabic{equation}}
\renewcommand{\thefigure}{\thechapter.\arabic{section}.\arabic{figure}}
\renewcommand{\thetable}{\thechapter.\arabic{section}.\arabic{table}}

% キャプション小技
\makeatletter
\newcommand{\figcaption}[1]{\def\@captype{figure}\caption{#1}}
\newcommand{\tblcaption}[1]{\def\@captype{table}\caption{#1}}
\makeatother

% 段落番号設定
\newcounter{parstep}[subsection]
\titleformat{\paragraph}[runin]{\normalfont}{\makebox[2zw][l]{(\arabic{parstep})}}{0pt}{}[\hspace{2zw}]
\titlespacing*{\paragraph}{0pt}{.6ex}{0pt}
\newcommand{\parstep}{\refstepcounter{parstep}\paragraph{\mbox{}}}

%================= 本文 =================
\begin{document}

% --- 表紙 ---
\pagestyle{empty}
\title{令和8年度 修士論文\\[2cm]題目\\[1cm]\huge{気中開閉器のアーク姿態が遮断成否に及ぼす影響}\\[5cm]}
\author{%
  埼玉大学大学院理工学研究科2年\\
  稲田研究室 24MM246 五十嵐柊登\\
  指導教員:稲田優貴 准教授\\[2cm]
  令和8年2月 日 提出}
\date{}
\maketitle
\clearpage

% --- 概要 ---
\newpage
\chapter*{概要}

ここに概要を記述

% --- 目次 ---
\newpage
\setcounter{tocdepth}{2}
\tableofcontents
\thispagestyle{empty}
\clearpage
\pagestyle{plain}
\setcounter{page}{1}

% --- 第1章 ---
\chapter{序論}
\section{研究背景・目的}
気中開閉器(PAS:Pole Air Switch)は,大気圧空気を消弧・絶縁媒体として用いた,
電流の遮断および投入を行う開閉装置である.
本開閉器は主に配電線路上に設置され,通常時には負荷電流の開閉を行うことで,
電力供給の信頼性を確保する役割を担っている\cite{kyuden_tandd}\cite{小田切司朗2009配電自動化の変遷}\cite{石谷卓也2012センサ付開閉器の負荷情報を活用した配電系統のオンライン構成変更}.
また,区分開閉器として配電線を複数の区間に分割することにより,
需要家側で発生した電気事故を区分する責任分界点として機能することで,
線路工事時や自然災害等による障害発生時において,
停電範囲を最小限に抑制することが可能である.
一般に,遮断器・開閉器の開閉は電極の機械的な接触および乖離によって行われており,
電流遮断時には接点の開極に伴って電極間にアーク放電が発生する.
このアーク放電は,冷却や拡散作用を受けるものの,発電機側から電力が時々刻々と注入されることで,
高温・高密度のプラズマ状態が維持され,その温度は数千〜数万Kに達する.
このため,開閉器の内部構造に熱的損傷を及ぼすおそれがあり,アークを確実かつ速やかに消弧し,電流を遮断することが求められる.
特に交流電源では,多くの場合,電源周波数の半周期ごとに電流零点を迎え,この瞬間にはアークプラズマへの注入電力が零となる.
この電流零点において,アークを消滅あるいは低導電性状態へと移行させる作用が十分に強ければ,電流を遮断することができる\cite{渋谷正豊1987第}.
一方で,アーク放電の遮断に失敗した場合には,爆発や持続的な火災を引き起こし,周辺機器に甚大な被害を与えるおそれがある.\cite{nowlen2008high}
したがって,迅速かつ確実にアークを消滅させる遮断機構は必要不可欠である.\\
 遮断動作をコンパクトな装置構成で実現するため,気中開閉器の内部には,多数の金属板(消弧板)が等間隔で積層された消弧室が設置されている\cite{freton2009overview}.
消弧板に加えてポリマー板が積層される場合もある.
開極時に引き伸ばされたアークが消弧室に侵入すると,積層された金属板によってアークは複数のセグメントに分割される\cite{mcbride2001review}.
この分割により,各接触点においてアノード・カソードスポットが形成され,それらの直列化に伴って電極降下電圧の合計値が増加し,アーク電圧が上昇する.
その結果,アーク抵抗が増大して高抵抗状態となり,導電性が低下するため,アーク分割は等価的にアーク冷却として作用する.
また,消弧室の壁材にはポリマー材料が多用されている.アークが消弧室の壁材と接触すると,ポリマーの溶発(アブレーション)が生じ,多量のアブレーションガスがアーク中に混入する\cite{freton2009overview}.
このガス混入過程は,等価的に分子性ガスの吹き付けとして機能し,分子の解離および電離に伴うエネルギー消費を通じて,アークに冷却効果をもたらす\cite{onchi2012effect,nakagawa2015numerical}.
さらに,アブレーションに伴い形成される複雑な流れ場は,アークの伸長や移動を促進し,冷却および消弧に寄与する.\\
 しかし,こうした消弧手段を講じた場合でも,電流ゼロ点においてアークが一度消弧した後に,再発弧する場合がある.
再発弧が発生すると,高温のアーク放電は次の電流ゼロ点まで持続することとなり,遮断完了は大きく遅延する.
さらに,再発弧が繰り返されると,最終的には電流遮断に失敗し,爆発や持続的な火災を引き起こして周辺機器に甚大な被害をもたらすおそれがある.
このため,再発弧の抑制は極めて重要な課題である.
再発弧の発生メカニズムには,これまでに二種類が報告されている\cite{hiraide2025two}.
一つ目は,電流ゼロ点直後に導電性の高い残留アークチャネルに過渡回復電圧が印加され,
わずかな残留電流が流れることで供給されるジュールエネルギーが,ガスの熱伝導および放射による冷却損失を上回ることで発生する,熱的再発弧である.
二つ目は,熱的再発弧が生じる時間領域の後,ギャップ内に残留する絶縁性ガスに対して,
ギャップの絶縁強度を上回る過渡回復電圧が印加されることにより絶縁破壊が生じることで発生する,誘電的再発弧である\cite{abb_livetank_2009}.
それぞれの再発弧における電流ゼロ区間および再発弧時における発光画像を,図1および図2にそれぞれ示す.
図\ref{fig:reignition}では,電流ゼロ区間においてアーク経路が残留し,高い導電性が維持されており,再発弧時には消弧前と同一の経路で再発弧しているのに対し,
図\ref{fig:reignition2}では電流ゼロ区間において経路の発光が途切れ,絶縁層が形成されている.さらに,再発弧時には消弧前とは異なる経路で再発弧が生じていることが分かる\cite{hiraide2025two}.
このように,電流ゼロ点付近に着目することは,再発弧の発生メカニズムの理解に有用である.\\
以上より,遮断成功に至るためには,熱的および誘電的再発弧の双方を防ぐ必要があり,とりわけ発生頻度が比較的高い,熱的再発弧の抑制が第一関門となる.
しかし,熱的再発弧の直接的な発生要因は未だ十分に解明されておらず,その原因究明は必須である.
そこで本研究では,熱的遮断の成否が決定づけられる電流ゼロ点付近において,分光画像センサーを用いてアーク温度および導電率の二次元分布を時間的連続性を有する形で取得し,
熱的再発弧の発生要因を特定することを目的とした.
\begin{figure}[htbp]
	\centering
	\begin{subfigure}[b]{0.48\columnwidth}
		\centering
		\includegraphics[width=0.8\linewidth]{netumae.png}
		\subcaption{電流ゼロ区間   }
		\label{fig:thermal}
	\end{subfigure}
	\hfill
	\begin{subfigure}[b]{0.48\columnwidth}
		\centering
		\includegraphics[width=0.8\linewidth]{netuato.png}
		\subcaption{再発弧直後   }
		\label{fig:dielectric}
	\end{subfigure}
	\caption{熱的再発弧時の発光画像}
	\label{fig:reignition}
\end{figure}

\begin{figure}[htbp]
	\centering
	\begin{subfigure}[b]{0.48\columnwidth}
		\centering
		\includegraphics[width=0.8\linewidth]{yuudenmae.png}
		\subcaption{電流ゼロ区間   }
		\label{fig:thermal2}
	\end{subfigure}
	\hfill
	\begin{subfigure}[b]{0.48\columnwidth}
		\centering
		\includegraphics[width=0.8\linewidth]{yuudenato.png}
		\subcaption{再発弧直後   }
		\label{fig:dielectric2}
	\end{subfigure}
	\caption{誘電的再発弧時の発光画像}
	\label{fig:reignition2}
\end{figure}
\newpage

% --- 第2章 ---
\chapter{原理}
\section{遮断器}
電力開閉装置は,電力系統の発電端から負荷端に至る巨大な電気回路のすべての節点(ノード)の入口あるいは出口に設置され,回路の開閉を行っている.
遮断器の型式は,油中で接点を開く方式,油の分解で発生する水素ガスをアークに吹きつける方式,
圧縮空気吹付方式,そして優れた遮断・絶縁性能を発揮するSF$_6$ガス吹付方式へと進展してきた.
現在,800kV級に至る基幹電力系統の遮断器は,そのほとんどがガス遮断器である.
また,遮断器のうち,配電変電所や受電設備に用いられる7.2kV以下の遮断器については,
油遮断器に始まり,磁気遮断器,ロータリーアークガス遮断器,さらに真空中で遮断を行う真空遮断器などが開発されてきた\cite{吉岡芳夫2006電力用遮断器の現状と将来}.
電力用開閉装置を,その保有する電流開閉能力によって大別すると,用途および性能に応じて以下のように分類される\cite{IEEJ1965}.\\
(1)
電路の接続切替えや断路を目的とし,無電流あるいはそれに近い状態で電路を開閉する断路器.\\
(2)
常時の負荷電流,あるいは過負荷電流程度までは安全に開閉できるが,負荷電流の数倍から数十倍に及ぶ故障電流の開閉を目的としない負荷遮断器,開閉器,接触器.\\
(3)
常時電流のみならず,故障電流をも支障なく開閉できるように設計された遮断器,あるいは電力ヒューズであり,特に保護用遮断器,保護用ヒューズとも呼ばれる.
これら表にすると\ref{tab:cb_class}のようになる\cite{IEICE1992}.
\begin{table}[tb]
  \centering
  \caption{開閉機器の比較}
  \label{tab:cb_class}
  \includegraphics[width=0.8\linewidth]{遮断器の表.png}
\end{table}
\FloatBarrier

\subsection{気中開閉器}
ここで気中開閉器の特徴を説明(さいげき、グリッド式とか)
\subsection{消弧室}
ここもあとで

\subsection{アーク放電}
\label{sec:arc_voltage}
電極間の金属接触が失われると,電極間を橋絡するようにアーク放電が発生し,電流はこれを通じて流れ続ける\cite{横水康伸1991ガス吹付けによる大電流のアーク遮断に関する基礎研究}.
図\ref{fig:arc_potential}にアーク放電における電位分布の模式図を示す\cite{Nakano1991}.
ここで,$V_\mathrm{c}$:陰極降下電圧,$V_\mathrm{a}$:陽極降下電圧,$V_\mathrm{col}$:アーク陽光柱電圧,$E_\mathrm{col}$:アーク電界,$L_\mathrm{a}$:アーク長とする.
陽光柱における電圧降下は,アーク電界とアーク長に比例する.
したがって,電流遮断時に両極間で発生するアーク電圧 $V_{\mathrm{arc}}$ は,
一般に次式で表される.

\begin{equation}
V_{\mathrm{arc}}
= V_\mathrm{a} + V_\mathrm{c} + V_{\mathrm{col}}
= V_{\mathrm{pol}} + E_{\mathrm{col}} L_\mathrm{a}
\label{eq:arc_voltage}
\end{equation}

ここで,陰極降下電圧と陽極降下電圧の和
\begin{equation}
V_{\mathrm{pol}} = V_\mathrm{c} + V_\mathrm{a}
\end{equation}
を陰・陽極降下電圧と定義する.

さらに,アークを直列に $n$ 個に分割した場合,アーク電圧は次式で与えられる\cite{bussiere2012electric}.

\begin{equation}
V_{\mathrm{arc}}
= n V_{\mathrm{pol}} + E_{\mathrm{col}} L_\mathrm{a}
\label{eq:multi_arc}
\end{equation}

電極材料や周囲気体条件に依存するものの,
陰極降下電圧$V_\mathrm{c}$はおよそ$1\,\mu\mathrm{m}$程度の領域で発生し,
その大きさは $10$~$20\,\mathrm{V}$ 程度である.
また,陽極降下電圧 $V_\mathrm{a}$も同様に約$1\,\mu\mathrm{m}$程度の範囲で発生し,
その大きさは数Vから$20\,\mathrm{V}$程度である\cite{IEEJ_Discharge_Handbook_1}.

一方,大電流条件下における動的アークの振る舞いは,
Cassieの動特性式によって次のように表される\cite{森田公1985交流電流零点前消弧ピークの発生条件に関する考察}.

\begin{equation}
\frac{1}{G}\frac{dG}{dt}
=
\frac{1}{\theta_M}
\left(
\frac{e^2}{e_0^2} - 1
\right)
\label{eq:cassie}
\end{equation}

ここで,
$G$はアークコンダクタンス [S/m],
$e$はアーク電圧 [V],
$e_0$は初期アーク電圧 [V],
$i$はアーク電流 [A] である.
また,$\theta_M$ はアーク時定数と呼ばれ,
アークコンダクタンスの時間応答特性を表すパラメータである.
アーク電圧を抵抗成分を含む回路の印加電圧より高めることでアークは消弧されるため,
アーク電圧の上昇は遮断器の性能向上に直結する.
式\ref{eq:arc_voltage}より,アーク電圧を高める手法として,
アーク電界を増大させること,
遮断部の直列分割数(直列遮断点数,発弧点数)を増加させることにより
陰・陽極降下電圧の合計値を増大させること,
および遮断部の長さを増大させることでアーク長を長くし,
陽光柱降下電圧を増大させることが挙げられる.
\begin{figure}[tb]
  \centering
  \includegraphics[width=0.6\linewidth]{アーク放電の電位分布.png}
  \caption{アーク放電の電位分布}
  \label{fig:arc_potential}
\end{figure}
\FloatBarrier

\subsection{アークの負性抵抗特性}
通常,オームの法則において電流は電圧に比例し,
その比例定数は抵抗として定義される.
しかし,アークの抵抗値は安定的ではなく,
アークの状態に応じて変化するため,
負性抵抗特性を示す場合がある.
図\ref{I_Vtokusei}に,アーク長を 2 mm とした場合の,
異なる圧力条件下における電流–電圧特性を示す.
図\ref{I_Vtokusei}より,いずれの圧力条件においても
低電流領域で負性抵抗特性を示していることが分かる.
特に,圧力が低い条件下では,
負性抵抗特性を示す電流領域が広くなる.

すなわち,大気圧条件下などの比較的低圧領域では,
電圧が上昇すると電流が減少する特性を示す.
また,\ref{sec:arc_voltage}節で前述したように,
アーク電圧はアーク長に比例して上昇するため,
実際にアーク長が変化する条件下では,
より強い負性抵抗特性が現れる.
この場合,アーク長が増大し,
すなわち電圧が上昇すると,
電流は大きく減少することになる\cite{菅泰雄1988高圧ヘリウム雰囲気中におけるアークの特性について}.

\begin{figure}[tb]
  \centering
  \includegraphics[width=0.6\linewidth]{電流電圧特性.png}
  \caption{異なる圧力下での電流電圧特性}
  \label{I_Vtokusei}
\end{figure}
\FloatBarrier

\subsection{他の遮断器の特徴}
気中開閉器との比較のため,本節では他方式の遮断器の特徴について述べる.

(1) 油遮断器 \cite{Mizushima1983}  
油中で接点を開離することにより,発生したアークが油を分解して生じる水素ガスの冷却作用,
局部的な圧力上昇,油の流動による置換作用,および油の高い絶縁耐力を利用してアークを消弧する方式である.
しかし,高速度での遮断が困難であることに加え,
絶縁油は劣化しやすく火災の原因となる場合もあるため,
保守・管理に多大な手間を要するなどの短所が多い.
このため,近年の日本ではほとんど使用されなくなっている.

(2) 空気遮断器  
10~30 気圧程度の圧縮空気を,電極間に発生したアークへ吹き付けることで,
アークを吹き飛ばし,電流零点通過時に急速に冷却することでイオンを消滅させ,
圧力によって消弧する遮断器である \cite{IEEJ_CB_1975}.
アーク時間が短く遮断性能が高いことに加え,
アークガスの除去が高速であるため高速度再投入が可能であるという特徴を有する.
一方で,遮断性能が圧縮空気の圧力,流速,流量などの条件に依存する点や,
アーク消弧時の空気流が機外へ排出され,
爆発的な動作音を発生させるという欠点もある \cite{Mizushima1983}.

(3) SF$_6$ガス遮断器  
空気遮断器において空気の代わりに,
絶縁性能の高い SF$_6$ ガスを消弧媒質として用いた遮断器である.
開発当初は,空気遮断器と同様に圧縮機によって予めガス圧力を高め,
遮断時に弁を開いて高圧ガスをアークに吹き付ける構造であったが,
現在ではアークの熱エネルギーを利用してガスを圧縮し,
吹付け圧力を高める構造が主流となっている \cite{Yanabu2000,吉岡芳夫2006電力用遮断器の現状と将来}.
SF$_6$ ガスは無毒・無臭・不燃で極めて安定な気体であり,
大気圧下において空気の約3倍という非常に優れた電気的絶縁性能を有することが最大の特徴である \cite{bo2017investigation}.
また,SF$_6$ ガスは平均自由行程が短く衝突電離を起こしにくいことに加え,
分子量が大きく電気的負性気体であるため,
電子を付着させて電界による加速を受けにくい負イオンを形成する.
これらの特性により,遮断器の遮断性能が向上する \cite{IEEJ_CB_1975,Hidaka2009}.
さらに,SF$_6$ ガス中のアークは空気中と比較してアーク温度が低く,
高温部が中心部に集中する特徴を有する \cite{IEEJ_CB_1975,Yanabu2000,Hidaka2009}.
このためアークの冷却が急速に進み,
電子密度の低下が早いことも遮断性能向上の一因である.
その結果,アーク中の導電率の減少が早く,
SF$_6$ ガス遮断器は高い遮断性能を示す.
現在,高電圧・大電力用遮断器の大部分は SF$_6$ ガス遮断器であるが \cite{吉岡芳夫2006電力用遮断器の現状と将来},
SF$_6$ ガスは地球温暖化係数が100年換算で CO$_2$ の 23,900 倍と非常に高く,
1997年の京都議定書において排出削減対象物質とされた.
このため,SF$_6$ ガスに代わる絶縁・消弧媒質に関する研究も進められている \cite{斉藤仁2009真空遮断器の最近の動向,大久保仁2003環境低負荷の真空遮断器}.

(4) 真空遮断器 \cite{IEEJ_CB_1975}  
真空遮断器は,真空中における高い絶縁耐力と,
金属蒸気や荷電粒子の拡散による優れた消弧特性を利用し,
真空容器内で電流の開閉および遮断を行う遮断器である.
主な利点として,
遮断動作を密閉容器内で行うため容器外にアークや高温ガスを放出しないこと,
可動軸の慣性が小さく多頻度開閉に適すること,
操作機構が小型で遮断器全体を小型・軽量化できること,
接触子の消耗が少なく真空バルブ寿命まで無保守・無点検で使用可能であること,
さらに接触部が完全密封構造であるため,
湿気,塵埃,有害ガスの影響を受けにくく,
安定した通電および遮断性能を維持できることが挙げられる.
また,SF$_6$ ガスのような環境負荷物質を消弧媒質として使用しないことから,
環境適合性が高く,近年注目を集めている.

\subsection{再発弧}
電気学会では,電流零点を迎えたアークに対し,
商用周波数の1/4未満の期間で再び電流が流れる場合を再発弧,
1/4以上の期間を経て再び電流が流れる場合を再点弧と定義している.

交流電流遮断は,通常電流の自然零点で行われ,
その後,電極間に印加される回路固有の過渡回復電圧に耐えることで遮断が完了する.
回復電圧の代表的な波形を図\ref{fig:電流零点における過渡回復電圧}に示す.

電流零点直後の十数μsの間は,
電極間に残留する導電性により,ごく微小な残留電流が流れる.
この期間において,冷却によるアークエネルギーの損失よりも,
再起電圧および残留電流によって注入されるエネルギーが大きくなる場合,再びプラズマが成長する.
この現象は熱的破壊と呼ばれる.
零点近傍におけるアークの時間的挙動は,Mayerの式により評価される.

\begin{equation}
\frac{1}{G}\frac{dG}{dt}
=
\frac{1}{\theta_M}
\left(
\frac{ei}{N}
-1
\right)
\label{eq:mayer}
\end{equation}

ただし,
$G$ はアークコンダクタンス [S/m],
$e$ はアーク電圧 [V],
$i$ はアーク電流 [A],
$N$ はアークの損失 [W],
$\theta_M$ はアーク時定数である.

一方,電極間の導電性が消滅している状態において,
絶縁耐力が十分に回復していない場合,
再起電圧による絶縁破壊によって再発弧が生じることがある.
この現象は誘電的破壊と呼ばれる.

誘電的破壊では,
過渡回復電圧により電子なだれが進展し,正イオン柱が形成される.
さらに,正イオンの空間電荷効果によって電子なだれが増幅され,ストリーマが発生する.
このストリーマが陽極側から伸展し,陰極に到達することで絶縁破壊に至る.

誘電的破壊が支配的な場合,
電流遮断が完了する時点で電圧が急激に上昇し,消弧ピークを示す.

\begin{figure}[tb]
  \centering
  \includegraphics[width=0.6\linewidth]{電流零点における過渡回復電圧.png}
  \caption{電流零点における過渡回復電圧}
  \label{fig:電流零点における過渡回復電圧}
\end{figure}
\FloatBarrier
\section{発光分光測定}
\subsection{発光分光法}
放電時に発生する発光は,種々のスペクトル成分の集合体であり,
回折格子を用いてこの光を分光することで,
各元素に固有の輝線スペクトルを取り出すことができる.
得られたスペクトルの強度や線幅を解析することにより,
プラズマの温度,電子密度,イオン温度などの物理量を推定できる.

発光分光測定の主な利点として,
マイクロ波法などの他の測定手法と比較して
時間的および空間的分解能に優れること,
ならびにプラズマに対して非干渉であり,
測定対象に影響を与えないことが挙げられる.
特に,非干渉性という特徴は,
高温プラズマや化学的に活性なプラズマの計測において,
プローブなどの測定器をプラズマ中に挿入することが困難な場合に,
極めて有効である \cite{YamamotoMurayama1995}.

気中アーク以外に発光分光測定が適用されている例として,
アーク溶接,遮断器内部におけるガスアーク,
および放電加工におけるアークプラズマなどが挙げられる.
平岡氏はアーク溶接において,
アルゴンおよびヘリウムの混合雰囲気中で発生するアークの
温度および電子密度の測定を行っている.
温度については,
Ar,Ar$^+$,He のスペクトル線を用いた
ボルツマンプロット法により評価しており,
電子密度については,
アルゴン・ヘリウム混合ガスに 1.5\% の水素を添加し,
水素 H$\beta$ 線のシュタルク広がりから算出している.
その結果,
電流 100 A において,
アーク温度は約 16\,000~18\,000 K,
電子密度は約
$1.1\times10^{22}$~$1.1\times10^{23}$ m$^{-3}$
($1.1\times10^{16}$~$1.1\times10^{17}$ cm$^{-3}$)
であることが示されている \cite{平岡和雄1993混合ガスシールドアークプラズマの発光分光特性とその解析}.

鹿野氏らは,
アルゴンガス遮断器内で発生するアーク放電に対して
発光分光測定を行い,
Ar および Ar$^+$ スペクトル線の線強度比法
(ボルツマンプロット法)を用いて
アーク温度を測定している.
測定の結果,
電流値 70~100 A の条件において,
アーク温度は約 12\,000~15\,000 K であることが報告されている\cite{鹿野竜大2019マイクロ秒分光計測を用いたアーク温度計測}.

\subsection{スペクトル線の発生機構}
プラズマによる発光メカニズムには,主に以下の三つがある.

(1) 自然放射  
原子やイオンが熱的に励起されると,
エネルギーの高い励起準位へ遷移する.
その後,寿命 $\tau$ をもって,
より低いエネルギー準位へ遷移する際に発光が生じる
\cite{YamamotoMurayama1995,DaidojiNakahara1996,KinoshitaOtaNagaiMinami2015}.
このような発光を自然放射と呼び,
原子・イオンに固有の線スペクトルを発生する
\cite{YamamotoMurayama1995,KinoshitaOtaNagaiMinami2015}.

(2) 再結合放射  
自由電子がイオンの束縛準位に捕獲される際,
電子の自由状態と束縛状態とのエネルギー差に対応した
発光が生じる.
この発光は再結合放射と呼ばれ,
連続スペクトルを形成する
\cite{YamamotoMurayama1995,KinoshitaOtaNagaiMinami2015}.

(3) 制動放射  
自由電子がイオンのクーロン力を受けて減速,
あるいは軌道を曲げられることで生じる発光であり,
制動放射と呼ばれる.
この発光は電子の運動エネルギーの連続的な変化に対応するため,
連続スペクトルを生じる
\cite{YamamotoMurayama1995,KinoshitaOtaNagaiMinami2015}.\\
一般に,自然放射によって得られるスペクトル線の放射強度は,
一方では与えられた振動数の放射を生じる量子的遷移確率により,
他方では対応する励起状態にある原子数によって決定される.
アークプラズマが熱平衡状態にあると仮定すると,
励起された原子はボルツマン分布に従う.
このとき,準位 $\mathrm{n}$ から $\mathrm{m}$ への遷移確率を $A_\mathrm{nm}$ とすると,
特定のアークスペクトル線の放射強度 $I$ は次式で与えられる.
\begin{equation}
I 
=
 A_\mathrm{nm} h \nu_\mathrm{n}
\frac{N_0 g_\mathrm{n}}{Z(T)}
\exp\!\left(-\frac{E_\mathrm{n}}{kT}\right)
\label{eq:line_intensity}
\end{equation}
ここで,
$g_\mathrm{n}$ は統計的重み,
$E_\mathrm{n}$ は励起エネルギー,
$h$ はプランク定数,
$\nu_\mathrm{n}$ は遷移に対応する振動数,
$N_0$ は原子密度,
$k$ はボルツマン定数,
$T$ はアーク温度,
$Z(T)$ は分配関数(状態和)である.

\subsection{局所熱平衡}
電子が電磁場から加速を受ける時間内において,
電子と重粒子が弾性衝突により十分なエネルギー交換を行っていれば,
電子の運動エネルギーは重粒子へと移行し,
電子温度$T_\mathrm{e}$と重粒子温度$T_\mathrm{h}$はほぼ等しい値をとる.
一方,十分なエネルギー交換が行われない場合には
$T_\mathrm{e} \neq T_\mathrm{h}$ となり,いわゆる熱的非平衡状態が現れる.
よく知られているように,電子温度$T_\mathrm{e}$ と重粒子温度$T_\mathrm{h}$の差を与える関係式は,
強い対流やドリフト,高い電子温度勾配,強い放射などの影響が無視できる場合,
定常・0 次元の電子エネルギー保存式,すなわち電子が電界から得るエネルギーが
重粒子との弾性衝突によって失われるエネルギーに等しいとする関係から導かれる\cite{田中康規2006熱プラズマにおける非平衡性}.
\begin{equation}
\frac{T_\mathrm{e} - T_\mathrm{h}}{T_\mathrm{e}}
=
\frac{3 \pi}{32} 
\left(
  \frac{e \, E \, \lambda_\mathrm{e}}{\frac{3}{2} \, k \, T_\mathrm{e}}
\right)^2
\frac{m_\mathrm{h}}{m_\mathrm{e}}
\label{eq:Te_Th}
\end{equation}

ここで,
$e$ は素電荷,
$m_e$ は電子の質量,
$m_h$ は重粒子の質量,
$\lambda_e$ は電子の平均自由行程,
$k$ はボルツマン定数,
$E$ は電界の大きさである.
式 \eqref{eq:Te_Th}の左辺$(T_\mathrm{e} - T_\mathrm{h})/T_\mathrm{e}$は,
電子と重粒子の温度差を表す無次元量であり,
熱的平衡性の指標として用いられる.
同式より,$T_\mathrm{e}$ と $T_\mathrm{h}$ が同程度の値となるためには,
電子が重粒子との衝突間に電界から得るエネルギー$\mathrm{e}E\lambda_\mathrm{e}$が,
電子のランダム運動(熱運動)エネルギー$\frac{3}{2}kT_\mathrm{e}$ に比べて十分小さいことが必要条件である.
そのためには,$T_\mathrm{e}$ が大きいこと,$E$ が小さいこと,$\lambda_\mathrm{e}$ が小さいことなどが必要である.
熱プラズマにおいては,
温度が約$10{,}000~\mathrm{K}$ と高く,
圧力も大気圧程度である場合が多い.
このような条件下では,
粒子間の衝突が十分に頻繁に起こるため,
局所熱平衡状態が成立すると考えられている\cite{1520853832517707008}.

\subsection{アーク温度測定法}
発光分光による温度決定法の代表的なものには\textit{Fowler Milne 法},\textit{二線強度比法},\textit{ボルツマンプロット法}が挙げられる.  
\textit{Fowler Milne 法}と\textit{二線強度比法}はイオン化・電離温度を,\textit{ボルツマンプロット法}は励起温度を決定する\cite{平岡和雄1998アークプラズマの分光計測}.
以下にそれぞれの特徴を示す.

(1) \textbf{Off axis 最大放射係数法} \\
 対象スペクトルの放射強度と温度を求めると,ある特定温度で最大の放射強度が得られる.  
最大強度に対する強度比から温度を特定する.  
遷移確率は不要であるが,局所熱平衡を仮定し,Sahaの電離平衡式を解くことが求められる.

(2) \textbf{二線強度比法}  \\
 異なる状態にある同原子粒子からの放射強度比により温度が決定される.  
熱平衡を仮定して Saha の電離式から粒子数 \(N\) を決定し,式 (\ref{eq:line_intensity}) の放射強度比と温度を対応づける.

(3) \textbf{相対強度比法 (ボルツマンプロット法)}  \\
 同状態にある粒子のスペクトル強度比から温度を導出する.  
Saha の電離式を使用しないが,粒子間でボルツマン分布が成立するという仮定を設ける必要がある.  
二準位をそれぞれ 1,2として,式(\ref{eq:line_intensity})の放射強度を計算する.  
放射強度比と温度の関係は次式のようになるため,温度が決定される\cite{平岡和雄1996各種分光法によるアークプラズマの温度評価}.

\begin{equation}
T
=
\frac{E_2 - E_1}
{k \ln
\left(
\frac{\nu_2 A_2 g_2 I_1}
{\nu_1 A_1 g_1 I_2}
\right)}
\label{eq:boltzmann_ratio}
\end{equation}


\subsection{空間粒子組成計算}
局所熱平衡状態 (\textit{Local Thermodynamic Equilibrium, LTE}) が成り立つ場合には,
温度と圧力が決まれば,熱力学第二法則に従って粒子組成状態は系のエントロピーが最大となるように決定される.  
具体的に熱平衡状態における粒子組成を計算するには,次の2つの熱統計力学的手法がよく用いられる.

(1) 系の Gibbs 自由エネルギー \(G\) を求め,これを最小化させるように粒子組成を求める.  

(2) 状態方程式のもと,解離平衡についての \textit{Guldberg-Waage の式} と電離平衡についての \textit{Saha の式} を連立させて各粒子密度を解く.

これらをいずれも状態方程式が成り立つという制約条件の下で解くことにより粒子組成が得られる.上記2つの方法は等価な関係である \cite{1520853832517707008}.

Gibbs の自由エネルギー \(G\) は次式で与えられる.

\[
G = \sum_{j=1}^{L} y_j \, 
\Biggl[
  \mu_j^0 
  + R_\mathrm{un} T \, 
  \ln \Biggl(
    \frac{y_j}{\sum_{p=1}^{L} y_p} 
    + \frac{P}{P_0}
  \Biggr)
\Biggr]
\]


ここで,  
\(y_j\):粒子 \(j\) のモル数 [mol],  
\(R = 8.31~\mathrm{J/mol/K}\):普遍気体定数,  
\(T\):温度 [K],  
\(L\):考慮する粒子種数,  
\(P\):圧力 [Pa],  
\(P_0\):標準圧力 (101325 Pa),  
\(\mu_j^0\):粒子 \(j\) の化学ポテンシャル [J/mol] であり,次式で与えられる.

\[
\mu_j^0
= - R_\mathrm{un} T \,
\ln \! \Biggl[
  \left( \frac{2 \pi m_j k T}{h^2} \right)^{3/2} 
  Z_j^\mathrm{int}(T) \frac{k T}{P_0}
\Biggr] 
+ \Delta H_{fj}
\]


分子 \( \mathrm{AB} \) と粒子 \( \mathrm{A}, \mathrm{B} \) との解離平衡に関する Guldberg-Waage の式は次式となる。

\[
\frac{n_{\mathrm{A}} \, n_{\mathrm{B}}}{n_{\mathrm{AB}}} 
=
\frac{(2 \pi m \, M_{\mathrm{AB}} k T)^{3/2}}{h^3} 
\frac{Z_{\mathrm{A}} Z_{\mathrm{B}}}{Z_{\mathrm{AB}}} 
\exp\Biggl( - \frac{E_{\mathrm{AB}}^\mathrm{dis}}{kT} \Biggr)
\]

ここで、  
\(n_{\mathrm{B}}, n_{\mathrm{AB}}\) はそれぞれ粒子 \( \mathrm{B}, \mathrm{AB} \) の数密度、  
\(Z_{\mathrm{B}}, Z_{\mathrm{AB}}\) はそれぞれ粒子 \( \mathrm{B}, \mathrm{AB} \) の分配関数(状態和)、  
\(E_{\mathrm{AB}}^\mathrm{dis}\) は粒子 \( \mathrm{AB} \) の解離エネルギー、  
\(M_{\mathrm{AB}}\) は換算質量 \(\displaystyle M_{\mathrm{AB}} = \frac{m_{\mathrm{A}} m_{\mathrm{B}}}{m_{\mathrm{A}} + m_{\mathrm{B}}}\) である。

次に、粒子 \( \mathrm{A} \) の電離平衡に関する Saha の式は次式となる。

\[
\frac{n_{\mathrm{A^+}} \, n_e}{n_{\mathrm{A}}} 
=
\frac{(2 \pi m_e k T)^{3/2}}{h^3} 
\frac{2 Z_{\mathrm{A^+}}}{Z_{\mathrm{A}}} 
\exp\Biggl( - \frac{E_{\mathrm{A}}^\mathrm{ion}}{kT} \Biggr)
\]

ここで、  
\(n_{\mathrm{A}}, n_{\mathrm{A^+}}, n_e\) はそれぞれ粒子 \( \mathrm{A}, \mathrm{A^+} \) および電子 \( e \) の数密度、  
\(Z_{\mathrm{A}}, Z_{\mathrm{A^+}}\) はそれぞれ粒子 \( \mathrm{A}, \mathrm{A^+} \) の分配関数、  
\(E_{\mathrm{A}}^\mathrm{ion}\) は粒子 \( \mathrm{A} \) の電離エネルギーである。

気体の状態方程式,プラズマとしての電気的中性条件,
元素比の式を考慮し,Newton-Raphson 法を基本とした連立方程式を解くことで粒子組成が得られる\cite{作田忠裕1978銅蒸気混入による高温空気中の電子密度の増大}.

\subsection{導電率算出}
Chapman-Enskog 法の第一近似を用いることにより、導電率は次式で表される。

\[
\sigma =
\frac{
  \dfrac{3 e^2}{16 k T} \, n_e 
  \left( \dfrac{2 \pi k T}{m_e} \right)^{1/2}
}{
  \sum\nolimits_{j=1}^{n}{}' n_j \, \pi \, \bar{\Omega}_{ej}^{(1,1)}
}
\]



ここで、  
\(e\):電気素量、  
\(k\):ボルツマン定数、  
\(T\):絶対温度、  
\(m_e\):電子の質量、  
\(n_e\):電子密度、  
\(n_j\):粒子 \(j\) の数密度、  
\(\bar{\Omega}_{ej}^{(1,1)}\):電子と粒子 \(j\) との間の衝突断面積(拡散断面積)、  
ここで、\(\sum\nolimits_{j=1}^{n}{}'\) は、和をとる際に
\(n_e \, \pi \, \bar{\Omega}_{ej}^{(1,1)}\) を除くことを意味する。


この式は、気中アークの銅蒸気を含む高温空気の輸送特性の解析や、アーク溶接の金属蒸気の挙動解析、酸素燃焼火炎中の電子伝導率の測定に用いられている。

\clearpage

% --- 第3章 ---
\chapter{実験手法}
\section{装置構成}

\clearpage

% --- 第4章 ---
\chapter{実験結果・考察}
\section{結果}

\clearpage

% --- 第5章 ---a
\chapter{結言}

\clearpage

% --- 謝辞 ---
\chapter*{謝辞}
\clearpage

% --- 参考文献 ---

\bibliographystyle{unsrtnat}
\bibliography{reference}

\end{document}