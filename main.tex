\documentclass[dvipdfmx,report,uplatex]{jsbook}

% ===== 基本レイアウト =====
\usepackage[a4paper,top=30mm,bottom=30mm,left=30mm,right=30mm]{geometry}
% ===== フォント(和文HaranoAji)=====
\usepackage[haranoaji]{pxchfon}

% ===== 数式・図表 =====
\usepackage[version=4]{mhchem}
\usepackage{amsmath,amssymb,amsfonts,bm}
\usepackage[dvipdfmx]{graphicx}
\usepackage{placeins}
\usepackage[dvipdfmx]{graphicx}
\usepackage{here}

% \graphicspath{{figures/}}
\usepackage[format=hang]{subcaption}
\usepackage{caption}
\captionsetup{labelsep=space}

% ===== 文献 =====
\usepackage[numbers,square]{natbib}

\setcitestyle{super}

% 単位(siunitx v3)
\usepackage{siunitx}
\sisetup{
  per-mode = power,
  propagate-math-font = true,
  reset-math-version = false,
  reset-text-family = false,
  reset-text-series = false,
  reset-text-shape = false,
  text-family-to-math = true,
  text-series-to-math = true,
  exponent-product = \times,
  output-decimal-marker = {.}
}
\sisetup{
  range-phrase = {\,\text{~}\,},
  range-units  = single
}

% ===== しおり・ハイパーリンク =====
\usepackage[
  dvipdfmx,
  setpagesize=false,
  bookmarks=true,
  bookmarksnumbered=true,
  bookmarksdepth=3,
  colorlinks=true,
  linkcolor = blue,
  pdfencoding=unicode,
  pdftitle={タイトル},
  pdfauthor={著者名},
  pdfsubject={サブジェクト},
  pdfkeywords={キーワード}
]{hyperref}
\usepackage{pxjahyper}


% ===== 見出し体裁 =====
\usepackage{titlesec}
\titleformat{\chapter}[hang]{\LARGE\gtfamily\bfseries}{第\arabic{chapter}章}{1em}{}
\titlespacing*{\chapter}{0pt}{0pt}{20pt}

% ===== 番号付け(章.節.連番)=====
\numberwithin{equation}{section}
\numberwithin{figure}{section}
\numberwithin{table}{section}
\renewcommand{\theequation}{\thechapter.\arabic{section}.\arabic{equation}}
\renewcommand{\thefigure}{\thechapter.\arabic{section}.\arabic{figure}}
\renewcommand{\thetable}{\thechapter.\arabic{section}.\arabic{table}}

% キャプション小技
\makeatletter
\newcommand{\figcaption}[1]{\def\@captype{figure}\caption{#1}}
\newcommand{\tblcaption}[1]{\def\@captype{table}\caption{#1}}
\makeatother

% 段落番号設定
\newcounter{parstep}[subsection]
\titleformat{\paragraph}[runin]{\normalfont}{\makebox[2zw][l]{(\arabic{parstep})}}{0pt}{}[\hspace{2zw}]
\titlespacing*{\paragraph}{0pt}{.6ex}{0pt}
\newcommand{\parstep}{\refstepcounter{parstep}\paragraph{\mbox{}}}

%================= 本文 =================
\begin{document}

% --- 表紙 ---
\pagestyle{empty}
\title{令和8年度 修士論文\\[2cm]題目\\[1cm]\huge{気中開閉器のアーク姿態が遮断成否に及ぼす影響}\\[5cm]}
\author{%
  埼玉大学大学院理工学研究科2年\\
  稲田研究室 24MM246 五十嵐柊登\\
  指導教員:稲田優貴 准教授\\[2cm]
  令和8年2月 日 提出}
\date{}
\maketitle
\clearpage

% --- 概要 ---
\newpage
\chapter*{概要}

 気中開閉器は,大気圧空気を消弧・絶縁媒体として用い,配電線路における電流の遮断および投入を担う機器である.
本開閉器は主に配電線路上に設置され,通常は負荷電流の開閉を行うことが主たる責務であるが,障害発生時には停電範囲を最小限に抑えることを目的とした区分開閉の責務も担う.

電流遮断時には,接点の開極に伴って高温アークが発生し,開閉器内部構造に熱的損傷を及ぼすおそれがある.そのため,アークを確実かつ速やかに消弧し,電流を遮断することが求められる.
気中開閉器の種類によっては,内部に多数の金属板を等間隔で積層した消弧室が設置されている.
アークが消弧室に接触すると,積層された消弧板によってアーク経路が分割され,アーク電圧の上昇に伴って導電性が低下する.
また,消弧板とともにポリマー板を積層する場合もあり,ポリマーの溶発(アブレーション)によってアークが冷却される.
しかし,これらの消弧手段を講じた場合であっても,電流ゼロ点においてアークが一度消弧した後に再発弧が生じ,遮断完了が大きく遅延することがある.
再発弧には,残留高温プラズマに起因する熱的再発弧と,絶縁破壊による誘電的再発弧が存在する.特に発生頻度の高い熱的再発弧の抑制が重要であるものの,その発生要因は十分に解明されていない.
そこで本研究では,熱的遮断の成否が決定される電流ゼロ点近傍における物理現象の解明を目的とし,分光画像センサーを用いたアーク温度および導電率の二次元分布測定を行った.
具体的には,Fe I 線を用いた二線強度比法によりアーク温度を算出し,空間粒子組成および Fe I–O I 線強度比に基づいて鉄蒸気混入率と導電率を算定した.
また,開極から第一電流ゼロ点までの時間を「ゼロ点時間」と定義し,ゼロ点時間ごとに条件を分類して評価を行った.

解析の結果,遮断成否を支配する要因として,以下の二つのメカニズムが明らかとなった.

第一に,ゼロ点時間が短いケースにおいて,可動電極上のアークスポット位置と遮断の成否には強い相関がみられた.
具体的には,遮断に成功したケースでは,スポット位置が可動電極上で接点部(極間)から遠ざかる傾向が顕著であった.
このとき,アークスポット近傍の導電率は,遮断が成功したケースで$10^{-1}$~$10^2\,\mathrm{S/m}$ と,遮断失敗ケースと比較して約1桁低い値を示した.
これは,スポット距離の増大によりアーク経路が強制的に伸長され,アーク抵抗の上昇に伴って導電率が抑制されたためであり,
その後、電流ゼロ点直後のジュール加熱が低減され、熱的遮断に至ったと推察される。
また,スポット距離が長くなることで,電極間に $4000\,\mathrm{K}$ 未満の低温領域の形成が促進され,誘電的遮断の能力も向上していると考えられる.
加えて,スポットが Cu-W 製の先端部から電極基部の Cu 材へ遷移したことで,材料間の融点や潜熱の差といった熱的特性の違いが,アーク温度の低下を助長した可能性も示唆された.

第二に,ゼロ点時間が長い条件下では,消弧室右上部に噴出するガスの発光強度が強いケースほど,再発弧が抑制される傾向が認められた.
噴出したガスは,発光強度が高いケースほど高温・高導電率な領域が広範囲に分布しており,それに伴いアーク経路の発光は低下していた.
これはアークの熱エネルギーが消弧部から効率的に排出されたためだと考えられる.
これを踏まえ,アーク経路の温度に対する噴出ガスの温度比を算出したところ,遮断成功例では再発弧例と比較して二倍以上の有意な差を示した.
なお,ガス噴出が比較的弱い場合であっても,アーク経路自体の発光が十分に抑制されていれば相対的に総和比は増大し,遮断は成功した.

以上の通り,本研究によって,気中開閉器における熱的遮断の成功には
「アークスポット移動に伴う経路の伸長」と「消弧室から空間への熱輸送によるアーク経路の冷却」という二つのプロセスが極めて重要であることが示された.


% --- 目次 ---
\newpage
\setcounter{tocdepth}{2}
\tableofcontents
\thispagestyle{empty}
\clearpage
\pagestyle{plain}
\setcounter{page}{1}

% --- 第1章 ---
\chapter{序論}
\section{研究背景・目的}
気中開閉器(PAS:Pole Air Switch)は,大気圧空気を消弧・絶縁媒体として用いた,
電流の遮断および投入を行う開閉装置である.
本開閉器は主に配電線路上に設置され,通常時には負荷電流の開閉を行うことで,
電力供給の信頼性を確保する役割を担っている\cite{kyuden_tandd}\cite{小田切司朗2009配電自動化の変遷}\cite{石谷卓也2012センサ付開閉器の負荷情報を活用した配電系統のオンライン構成変更}.
また,区分開閉器として配電線を複数の区間に分割し,需要家側で発生した電気事故を区分する責任分界点として機能する.
これにより,線路工事時や自然災害等による障害発生時においても,停電範囲を最小限に抑制することが可能である.
一般に,遮断器・開閉器の開閉は電極の機械的な接触および乖離によって行われており,
電流遮断時には,接点の開極に伴いアーク放電が発生する.
このアーク放電は,冷却や拡散作用を受けるものの,発電機側から電力が時々刻々と注入されることで,
高温・高密度のプラズマ状態が維持され,その温度は数千〜数万Kに達する.
このため,開閉器の内部構造に熱的損傷を及ぼすおそれがあり,アークを確実かつ速やかに消弧し,電流を遮断することが求められる.
特に交流電源では,多くの場合,電源周波数の半周期ごとに電流零点を迎え,この瞬間にはアークプラズマへの注入電力が零となる.
この電流零点において,アークを消滅あるいは低導電性状態へと移行させる作用が十分に強ければ,電流を遮断することができる\cite{渋谷正豊1987第}.
一方で,アーク放電の遮断に失敗した場合には,爆発や持続的な火災を引き起こし,周辺機器に甚大な被害を与えるおそれがある.\cite{nowlen2008high}
したがって,迅速かつ確実にアークを消滅させる遮断機構は必要不可欠である.\\
 遮断動作をコンパクトな装置構成で実現するため,気中開閉器の内部には,多数の金属板(消弧板)が等間隔で積層された消弧室が設置されている\cite{freton2009overview}.
消弧板に加えてポリマー板が積層される場合もある.
開極時に引き伸ばされたアークが消弧室に侵入すると,積層された金属板によってアークは複数のセグメントに分割される\cite{mcbride2001review}.
この分割により,各接触点においてアノード・カソードスポットが形成され,それらの直列化に伴って電極降下電圧の合計値が増加し,アーク電圧が上昇する.
その結果,アーク抵抗が増大して高抵抗状態となり,導電性が低下するため,アーク分割は等価的にアーク冷却として作用する.
また,消弧室の壁材にはポリマー材料が多用されている.アークが消弧室の壁材と接触すると,ポリマーの溶発(アブレーション)が生じ,多量のアブレーションガスがアーク中に混入する\cite{freton2009overview}.
このガス混入過程は,等価的に分子性ガスの吹き付けとして機能し,分子の解離および電離に伴うエネルギー消費を通じて,アークに冷却効果をもたらす\cite{onchi2012effect,nakagawa2015numerical}.
さらに,アブレーションに伴い形成される複雑な流れ場は,アークの伸長や移動を促進し,冷却および消弧に寄与する.\\
 しかし,こうした消弧手段を講じた場合でも,電流ゼロ点においてアークが一度消弧した後に,再発弧する場合がある.
再発弧が発生すると,高温のアーク放電が次の電流ゼロ点まで持続することとなり,遮断完了は大きく遅延する.
さらに,再発弧が繰り返されると,最終的には電流遮断に失敗し,爆発や持続的な火災を引き起こして周辺機器に甚大な被害をもたらすおそれがある.
このため,再発弧の抑制は極めて重要な課題である.
再発弧の発生メカニズムには,これまでに二種類が報告されている\cite{hiraide2025two}.
一つ目は,電流ゼロ点直後に導電性の高い残留アークチャネルに過渡回復電圧が印加され,
わずかな残留電流が流れることで供給されるジュールエネルギーが,ガスの熱伝導および放射による冷却損失を上回ることで発生する,熱的再発弧である.
二つ目は,熱的再発弧が生じる時間領域の後,ギャップ内に残留する絶縁性ガスに対して,
ギャップの絶縁強度を上回る過渡回復電圧が印加されることにより絶縁破壊が生じることで発生する,誘電的再発弧である\cite{abb_livetank_2009}.
それぞれの再発弧における電流ゼロ区間および再発弧時における発光画像を,図\ref{fig:reignition}および図\ref{fig:reignition2}にそれぞれ示す.
図\ref{fig:reignition}に示す熱的再発弧では,電流ゼロ区間において,アーク経路が残留することで高い導電性が維持されており,再発弧時には消弧前と同一の経路で再発弧している.
一方,図\ref{fig:reignition2}に示す誘電的再発弧では,電流ゼロ区間において経路の発光が途切れ,絶縁層が形成されている.さらに,再発弧時には消弧前とは異なる経路で再発弧が生じている\cite{hiraide2025two}.
このように,電流零点近傍の現象を詳細に調査することは,再発弧メカニズムの解明において極めて有用である.\\
 遮断成功に至るためには,熱的および誘電的再発弧の双方を防ぐ必要があり,とりわけ発生頻度が比較的高い,熱的再発弧の抑制が第一関門となる.
しかし,熱的再発弧の直接的な発生要因は未だ十分に解明されておらず,その原因究明は必須である.
そこで本研究では,熱的遮断の成否が決定づけられる電流ゼロ点付近において,分光画像センサーを用いてアーク温度および導電率の二次元分布を時間的連続性をもって取得し,
熱的再発弧の発生要因を特定することを目的とした.
\begin{figure}[htbp]
	\centering
	\begin{subfigure}[b]{0.48\columnwidth}
		\centering
		\includegraphics[width=0.7\linewidth]{netumae.png}
		\subcaption{ 電流ゼロ区間}
		\label{fig:thermal}
	\end{subfigure}
	\hfill
	\begin{subfigure}[b]{0.48\columnwidth}
		\centering
		\includegraphics[width=0.7\linewidth]{netuato.png}
		\subcaption{ 再発弧直後}
		\label{fig:dielectric}
	\end{subfigure}
	\caption{熱的再発弧時の発光画像}
	\label{fig:reignition}
\end{figure}

\begin{figure}[htbp]
	\centering
	\begin{subfigure}[b]{0.48\columnwidth}
		\centering
		\includegraphics[width=0.7\linewidth]{yuudenmae.png}
		\subcaption{ 電流ゼロ区間}
		\label{fig:thermal2}
	\end{subfigure}
	\hfill
	\begin{subfigure}[b]{0.48\columnwidth}
		\centering
		\includegraphics[width=0.7\linewidth]{yuudenato.png}
		\subcaption{ 再発弧直後}
		\label{fig:dielectric2}
	\end{subfigure}
	\caption{誘電的再発弧時の発光画像}
	\label{fig:reignition2}
\end{figure}
\newpage

% --- 第2章 ---
\chapter{原理}
\section{遮断器}
電力開閉装置は,電力系統の発電端から負荷端に至る巨大な電気回路網の各節点(ノード)に設置され,回路の開閉制御を担っている.
遮断器の形式は,油中で接点を開離する方式から,油の熱分解により発生する水素ガスをアークに吹き付ける方式,圧縮空気吹き付け方式を経て,優れた遮断・絶縁性能を有する $\mathrm{SF}_6$ ガス吹き付け方式へと発展を遂げてきた.
現在,$800\,\mathrm{kV}$ 級に至る基幹電力系統の遮断器は,そのほとんどがガス遮断器(GCB)である.
また,配電用変電所や受電設備に用いられる $7.2\,\mathrm{kV}$ 以下の電圧階級においては,油遮断器に始まり,磁気遮断器,ロータリーアークガス遮断器,さらに真空中で遮断を行う真空遮断器(VCB)などが広く普及している\cite{吉岡芳夫2006電力用遮断器の現状と将来}.

電力用開閉装置は,その電流開閉能力に基づき,用途および性能に応じて以下のように大別される\cite{IEEJ1965}.

\begin{enumerate}
    \item \textbf{断路器}:電路の接続切り替えや保守時の隔離を目的とし,無電流あるいはそれに極めて近い状態で電路を開閉する.
    \item \textbf{負荷開閉器}:定格負荷電流や過負荷電流程度の開閉は安全に行えるが,短絡事故等に伴う故障電流の遮断能力は有さない.
    \item \textbf{遮断器}:常時電流のみならず,過酷な故障電流をも支障なく遮断できるように設計されており,系統の保護用装置として機能する.
\end{enumerate}

これらの比較をまとめると表\ref{tab:cb_class}のようになる\cite{IEICE1992}.

\begin{table}[H]
  \centering
  \caption{開閉機器の比較}
  \label{tab:cb_class}
  \includegraphics[width=0.7\linewidth]{遮断器の表.png}
\end{table}

\subsection{アーク放電}
\label{sec:arc_voltage}

電極間の金属接触が断たれると,両極間を橋絡するようにアーク放電が発生し,電流はこれを通じて継続して流れる\cite{横水康伸1991ガス吹付けによる大電流のアーク遮断に関する基礎研究}.
図\ref{fig:arc_potential}に,アーク放電における電位分布の模式図を示す\cite{Nakano1991}.
ここで,$V_\mathrm{c}$ は陰極降下電圧,$V_\mathrm{a}$ は陽極降下電圧,$V_\mathrm{col}$ はアーク陽光柱電圧,$E_\mathrm{col}$ はアーク電界,$L_\mathrm{a}$ はアーク長を表す.
陽光柱における電位降下は,アーク電界とアーク長に比例する.

したがって,電流遮断時に両極間で発生するアーク電圧 $V_{\mathrm{arc}}$ は,一般に次式で定義される.
\begin{equation}
V_{\mathrm{arc}} = V_\mathrm{a} + V_\mathrm{c} + V_{\mathrm{col}}
= V_{\mathrm{pol}} + E_{\mathrm{col}} L_\mathrm{a}
\label{eq:arc_voltage}
\end{equation}
ここで,陰極降下電圧と陽極降下電圧の和である $V_{\mathrm{pol}}$ は,次式で与えられる.
\begin{equation}
V_{\mathrm{pol}} = V_\mathrm{c} + V_\mathrm{a}
\end{equation}

さらに,アークを直列に $n$ 個に分割した場合,アーク電圧は次式のように記述される\cite{bussiere2012electric}.
\begin{equation}
V_{\mathrm{arc}} = n V_{\mathrm{pol}} + E_{\mathrm{col}} L_\mathrm{a}
\label{eq:multi_arc}
\end{equation}

電極材料や周囲の気体条件に依存するものの,陰極降下電圧 $V_\mathrm{c}$ はおよそ \qty{1}{\micro\meter} 程度の極めて狭い領域で発生し,その大きさは \qtyrange{10}{20}{\volt} 程度である.
また,陽極降下電圧 $V_\mathrm{a}$ も同様に約 \qty{1}{\micro\meter} 程度の範囲で発生し,その値は数 \unit{\volt} から \qty{20}{\volt} 程度となる\cite{IEEJ_Discharge_Handbook_1}.

一方,大電流条件下における動的アークの振る舞いは,Cassieの動特性式によって次式のように記述される\cite{森田公1985交流電流零点前消弧ピークの発生条件に関する考察}.
\begin{equation}
\frac{1}{G}\frac{dG}{dt}
= \frac{1}{\theta_M}
\left(
\frac{e^2}{e_0^2} - 1
\right)
\label{eq:cassie}
\end{equation}
ここで,$G$ はアークコンダクタンス \unit{\siemens\per\meter},$e$ はアーク電圧 \unit{\volt},$e_0$ は定常アーク電圧\unit{\volt} である.
また,$\theta_M$ はアーク時定数と呼ばれ,アークコンダクタンスの時間応答特性を支配するパラメータである.

遮断器においては,アーク電圧を系統の印加電圧よりも高めることで消弧を達成するため,アーク電圧の増大は遮断性能の向上に直結する.
式\ref{eq:arc_voltage}に基づけば,アーク電圧を高める具体的な手法として,
(i) 強制冷却等によるアーク電界の増大,(ii) 遮断部の多点直列化(直列遮断点数や発弧点数の増加)による陰・陽極降下電圧の総計の増大,
および (iii) 遮断部構造の長尺化によるアーク長の増大が挙げられる.
  \begin{figure}
  \centering
  \includegraphics[width=0.6\linewidth]{アーク放電の電位分布.png}
  \caption{アーク放電の電位分布}
  \label{fig:arc_potential}
\end{figure}

\FloatBarrier
\subsection{アークの負性抵抗特性}
通常,オームの法則において電流は電圧に比例し,その比例定数の逆数は抵抗として定義される.
しかし,アークの抵抗値は一定ではなく,アークの状態(温度や電離度など)に応じて変化するため,負性抵抗特性を示す場合がある.
図\ref{I_Vtokusei}に,アーク長を2 mmとした場合の,異なる圧力条件下における電流–電圧特性を示す.
図より,いずれの圧力条件においても,特に低電流領域で電圧の増加に伴い電流が減少する負性抵抗特性が確認できる.
この特性は,圧力が低い条件下ほど,負性抵抗を示す電流領域が広くなる傾向にある.
すなわち,大気圧付近を含む比較的低圧な領域では,電圧の上昇に対して電流が減少するという挙動を示す.
また,\ref{sec:arc_voltage}節で述べたように,アーク電圧はアーク長に比例して上昇する.
したがって,アーク長が動的に変化する条件下では,より顕著な負性抵抗特性が現れる.
この場合,アーク長の増大(=電圧の上昇)に伴い,電流は大幅に減少することになる\cite{菅泰雄1988高圧ヘリウム雰囲気中におけるアークの特性について}.

\begin{figure}[tb]
  \centering
  \includegraphics[width=0.6\linewidth]{電流電圧特性.png}
  \caption{異なる圧力下での電流電圧特性}
  \label{I_Vtokusei}
\end{figure}
\FloatBarrier


\subsection{他の遮断器の特徴}
気中開閉器との比較のため,本節では他方式の遮断器の特徴について述べる.

(1) 油遮断器 \cite{Mizushima1983}  

油遮断器は,油中で接点を開離することにより発生したアークが油を分解して生じる水素ガスの冷却作用,
局部的な圧力上昇,油の流動による置換作用,および油の高い絶縁耐力を利用してアークを消弧する方式である.
しかし,高速度での遮断が困難であることに加え,
絶縁油は劣化しやすく火災の原因となる場合もあるため,
保守・管理に多大な手間を要するという欠点を有する.
このため,近年の日本ではほとんど使用されなくなっている.\\

(2) 空気遮断器  

空気遮断器は,10~30 気圧程度の圧縮空気を電極間に発生したアークへ吹き付けることにより,
アークを吹き飛ばすとともに,電流零点通過時に急速に冷却することでイオンを消滅させ,
圧力によって消弧する遮断器である \cite{IEEJ_CB_1975}.
アーク時間が短く遮断性能が高いことに加え,
アークガスの除去が高速であるため,高速度再投入が可能であるという特徴を有する.
一方で,遮断性能が圧縮空気の圧力,流速,流量などの条件に強く依存する点や,
アーク消弧時の空気流が機外へ排出され,
爆発的な動作音を発生させるという欠点も指摘されている \cite{Mizushima1983}.\\

(3) SF$_6$ ガス遮断器  

SF$_6$ ガス遮断器は,空気遮断器において空気の代わりに,
絶縁性能の高い SF$_6$ ガスを消弧媒質として用いた遮断器である.
開発当初は,圧縮機によって予めガス圧力を高め,
遮断時に弁を開いて高圧ガスをアークに吹き付ける構造であったが,
現在ではアークの熱エネルギーを利用してガスを圧縮し,
吹付け圧力を高める構造が主流となっている
\cite{Yanabu2000,吉岡芳夫2006電力用遮断器の現状と将来}.
SF$_6$ ガスは無毒・無臭・不燃で極めて安定な気体であり,
大気圧下において空気の約3倍という非常に高い電気的絶縁性能を有することが最大の特徴である\\
\cite{bo2017investigation}.
また,平均自由行程が短く衝突電離を起こしにくいことに加え,
分子量が大きく電気的負性気体であるため,
電子を付着させて電界による加速を受けにくい負イオンを形成する.
これらの特性により,遮断器の遮断性能が大きく向上する
\cite{IEEJ_CB_1975,Hidaka2009}.
さらに,SF$_6$ ガス中のアークは空気中と比較してアーク温度が低く,
高温部が中心部に集中する特徴を有する
\cite{IEEJ_CB_1975,Yanabu2000,Hidaka2009}.
このためアークの冷却が急速に進み,
電子密度の低下が早いことも遮断性能向上の一因である.
その結果,アーク中の導電率の減少が早く,
SF$_6$ ガス遮断器は高い遮断性能を示す.
現在,高電圧・大電力用遮断器の大部分は SF$_6$ ガス遮断器であるが
\cite{吉岡芳夫2006電力用遮断器の現状と将来},
SF$_6$ ガスは地球温暖化係数が100年換算で CO$_2$ の 23,900 倍と非常に高く,
1997年の京都議定書において排出削減対象物質とされた.
このため,SF$_6$ ガスに代わる絶縁・消弧媒質に関する研究も進められている
\cite{斉藤仁2009真空遮断器の最近の動向,大久保仁2003環境低負荷の真空遮断器}.

(4) 真空遮断器 \cite{IEEJ_CB_1975}  

真空遮断器は,真空中における高い絶縁耐力と,
金属蒸気や荷電粒子の拡散による優れた消弧特性を利用し,
真空容器内で電流の開閉および遮断を行う遮断器である.
主な利点として,
遮断動作を密閉容器内で行うため容器外にアークや高温ガスを放出しないこと,
可動部の慣性が小さく多頻度開閉に適すること,
操作機構が小型で遮断器全体を小型・軽量化できること,
接触子の消耗が少なく真空バルブ寿命まで無保守・無点検で使用可能であること,
さらに接触部が完全密封構造であるため,
湿気,塵埃,有害ガスの影響を受けにくく,
安定した通電および遮断性能を維持できることが挙げられる.
また,SF$_6$ ガスのような環境負荷物質を消弧媒質として使用しないことから,
環境適合性が高く,近年注目を集めている.

\subsection{再発弧}
電気学会では,電流零点を迎えたアークに対し,
商用周波数の1/4未満の期間で再び電流が流れる場合を再発弧,
1/4以上の期間を経て再び電流が流れる場合を再点弧と定義している.

交流電流遮断は,通常電流の自然零点で行われ,
その後,電極間に印加される回路固有の過渡回復電圧に耐えることで遮断が完了する.
回復電圧の代表的な波形を図\ref{fig:電流零点における過渡回復電圧}に示す.

電流零点直後の十数μsの間は,
電極間に残留する導電性により,ごく微小な残留電流が流れる.
この期間において,冷却によるアークエネルギーの損失よりも,
再起電圧および残留電流によって注入されるエネルギーが大きくなる場合,再びプラズマが成長する.
この現象は熱的破壊と呼ばれる.
零点近傍におけるアークの時間的挙動は,Mayerの式により評価される.

\begin{equation}
\frac{1}{G}\frac{dG}{dt}
=
\frac{1}{\theta_M}
\left(
\frac{ei}{N}
-1
\right)
\label{eq:mayer}
\end{equation}

ただし,
$G$ はアークコンダクタンス [S/m],
$e$ はアーク電圧 [V],
$i$ はアーク電流 [A],
$N$ はアークの損失 [W],
$\theta_M$ はアーク時定数である.

一方,電極間の導電性が消滅している状態において,
絶縁耐力が十分に回復していない場合,
再起電圧による絶縁破壊によって再発弧が生じることがある.
この現象は誘電的破壊と呼ばれる.

誘電的破壊では,
過渡回復電圧により電子なだれが進展し,正イオン柱が形成される.
さらに,正イオンの空間電荷効果によって電子なだれが増幅され,ストリーマが発生する.
このストリーマが陽極側から伸展し,陰極に到達することで絶縁破壊に至る.

誘電的破壊が支配的な場合,
電流遮断が完了する時点で電圧が急激に上昇し,消弧ピークを示す.

\begin{figure}[tb]
  \centering
  \includegraphics[width=0.6\linewidth]{電流零点における過渡回復電圧.png}
  \caption{電流零点における過渡回復電圧}
  \label{fig:電流零点における過渡回復電圧}
\end{figure}
\FloatBarrier
\clearpage
\section{発光分光測定}
\subsection{発光分光法}

放電時に発生する発光は,
種々のスペクトル成分から成る集合体である.
回折格子を用いてこの光を分光することで,
各元素(原子・イオン)に固有の輝線スペクトルを得ることができる.
得られたスペクトルの強度,強度比,あるいは線幅を解析することにより,
プラズマの電子温度,電子密度,ガス温度などの物理量を推定することが可能である.

発光分光測定の主な利点として,
マイクロ波法等の他の測定手法と比較して
時間的および空間的分解能に優れること,
ならびにプラズマに対して非摂動(非干渉)であり,
測定対象の状態を乱さないことが挙げられる.
特にこの非摂動性は,
高温プラズマや化学的に活性なプラズマにおいて,
静電プローブ等の測定器を内部に挿入することが
物理的に困難な場合に極めて有効である
\cite{YamamotoMurayama1995}.

気中アーク以外に発光分光測定が適用されている例として,
アーク溶接,遮断器内のガスアーク,
および放電加工におけるアークプラズマなどが挙げられる.
平岡はアーク溶接において,
アルゴン(Ar)およびヘリウム(He)の混合雰囲気中で発生するアークを対象に,
温度および電子密度の測定を行っている.
温度については,
Ar,Ar$^{+}$,He のスペクトル線を用いた
ボルツマンプロット法により評価しており,
電子密度については,
混合ガスに 1.5\% の水素を添加し,
水素 H$_{\beta}$ 線のシュタルク広がり
(Stark broadening)から算出している.
その結果,
電流 100 A において,
アーク温度は約 16,000~18,000 K,
電子密度は約
$1.1\times10^{22}$~$1.1\times10^{23}$ m$^{-3}$
($1.1\times10^{16}$~$1.1\times10^{17}$ cm$^{-3}$)
であることが示されている
\cite{平岡和雄1993混合ガスシールドアークプラズマの発光分光特性とその解析}.

また,鹿野らは,
アルゴンガス遮断器内で発生するアーク放電に対して分光測定を行い,
Ar および Ar$^{+}$ の輝線強度比法
(ボルツマンプロット法)を用いて
アーク温度を算出している.
その結果,
電流値 70~100 A の条件において,
アーク温度は約 12,000~15,000 K であることが
報告されている
\cite{鹿野竜大2019マイクロ秒分光計測を用いたアーク温度計測}.

\subsection{スペクトル線の発生機構}
プラズマによる発光メカニズムは,主に以下の三つの過程に分類される.\\

(1) \textbf{自然放射(自発放出)}  

原子やイオンが熱的に励起されて高いエネルギー準位へ遷移した後,固有の寿命 $\tau$ を経て,より低いエネルギー準位へ遷移する際に光を放出する過程である
\cite{YamamotoMurayama1995,DaidojiNakahara1996,KinoshitaOtaNagaiMinami2015}.
この過程により,原子やイオンの種類に固有の輝線スペクトルが形成される
\cite{YamamotoMurayama1995,KinoshitaOtaNagaiMinami2015}.\\

(2) \textbf{再結合放射}  

自由電子がイオンの束縛準位に捕獲される際,電子の自由状態(運動エネルギー)と束縛状態とのエネルギー差に相当する光を放出する現象である.
この発光は再結合放射と呼ばれ,電子の運動エネルギーが連続的に分布しているため,連続スペクトルを形成する
\cite{YamamotoMurayama1995,KinoshitaOtaNagaiMinami2015}.\\

(3) \textbf{制動放射}  

自由電子がイオンのクーロン力を受けて減速,あるいは進路を変えられる際に生じる発光である.
この場合も電子の運動エネルギーが連続的に変化するため,連続スペクトルが観測される
\cite{YamamotoMurayama1995,KinoshitaOtaNagaiMinami2015}.

一般に,自然放射によるスペクトル線の放射強度は,量子力学的な遷移確率と,
対応する励起状態に存在する粒子数によって決定される.
アークプラズマが局所熱平衡(LTE)状態にあると仮定すると,
励起原子数はボルツマン分布に従う.
このとき,準位 $n$ から $m$ への遷移確率を $A_{nm}$ とすると,
特定のスペクトル線の放射強度 $I$ は次式で与えられる.

\begin{equation}
I = \frac{A_{nm} h \nu_{nm} N_0 g_n}{Z(T)}
\exp \left( -\frac{E_n}{kT} \right)
\label{eq:line_intensity}
\end{equation}

ここで,$g_n$ は統計的重み,
$E_n$ は励起エネルギー [J],
$h$ はプランク定数 [J$\cdot$s],
$\nu_{nm}$ は遷移に対応する振動数 [Hz],
$N_0$ は基底状態の原子密度 [m$^{-3}$],
$k$ はボルツマン定数 [J/K],
$T$ はアーク温度 [K],
$Z(T)$ は分配関数(状態和)である.



\subsection{局所熱平衡}
電子が電磁場から加速を受ける時間内において,
電子と重粒子が弾性衝突により十分なエネルギー交換を行っていれば,
電子の運動エネルギーは重粒子へ移行し,
電子温度 $T_{\mathrm{e}}$ と重粒子温度 $T_{\mathrm{h}}$ は
ほぼ等しい値をとる.
一方,エネルギー交換が十分に行われない場合には
$T_{\mathrm{e}} \neq T_{\mathrm{h}}$ となり,
いわゆる熱的非平衡状態が生じる.

一般に,電子温度 $T_{\mathrm{e}}$ と重粒子温度 $T_{\mathrm{h}}$ の差を与える関係式は,
強い対流やドリフト,急峻な温度勾配,および強い放射の影響が無視できる場合,
定常かつ 0 次元近似の電子エネルギー保存式から導かれる.
すなわち,電子が電界から得るエネルギーと,
重粒子との弾性衝突によって失われるエネルギーが等しいとする
エネルギー収支関係より,次式が得られる
\cite{田中康規2006熱プラズマにおける非平衡性}.

\begin{equation}
\frac{T_{\mathrm{e}} - T_{\mathrm{h}}}{T_{\mathrm{e}}}
=
\frac{3 \pi}{32}
\left(
\frac{e E \lambda_{\mathrm{e}}}{\frac{3}{2} k T_{\mathrm{e}}}
\right)^2
\frac{m_{\mathrm{h}}}{m_{\mathrm{e}}}
\label{eq:Te_Th}
\end{equation}

ここで,$e$ は素電荷 [C],
$m_{\mathrm{e}}$ は電子の質量 [kg],
$m_{\mathrm{h}}$ は重粒子の質量 [kg],
$\lambda_{\mathrm{e}}$ は電子の平均自由行程 [m],
$k$ はボルツマン定数 [J/K],
$E$ は電界の強さ [V/m] である.
式 \eqref{eq:Te_Th} の左辺は,
電子と重粒子の温度差を規格化した無次元量であり,
熱的平衡性の指標として用いられる.

同式より,$T_{\mathrm{e}}$ と $T_{\mathrm{h}}$ が近似的に等しくなるためには,
電子が平均自由行程 $\lambda_{\mathrm{e}}$ の間に
電界から得るエネルギー $e E \lambda_{\mathrm{e}}$ が,
電子の熱運動エネルギー $\frac{3}{2} k T_{\mathrm{e}}$
に比べて十分に小さいことが必要条件となる.
したがって,局所熱平衡が成立するためには,
$T_{\mathrm{e}}$ が十分に高いこと,
電界 $E$ が小さいこと,
あるいは粒子密度が高く
$\lambda_{\mathrm{e}}$ が小さいことなどが求められる.

大気圧下のアークプラズマ等の熱プラズマにおいては,
温度が $10{,}000\,\mathrm{K}$ 程度と高く,
かつ粒子密度も高いため,
粒子間の衝突が極めて頻繁に生じる.
このような条件下ではエネルギー交換が十分に行われ,
局所熱平衡(LTE)状態が概ね成立すると考えられている
\cite{1520853832517707008}.


\subsection{アーク温度測定法}
発光分光による温度決定法の代表的なものには\textit{Fowler Milne 法},\textit{二線強度比法},\textit{ボルツマンプロット法}が挙げられる.  
\textit{Fowler Milne 法}と\textit{二線強度比法}はイオン化・電離温度を,\textit{ボルツマンプロット法}は励起温度を決定する\cite{平岡和雄1998アークプラズマの分光計測}.
以下にそれぞれの特徴を示す.

(1) \textbf{Off axis 最大放射係数法} \\
 対象スペクトルの放射強度と温度を求めると,ある特定温度で最大の放射強度が得られる.  
最大強度に対する強度比から温度を特定する.  
遷移確率は不要であるが,局所熱平衡を仮定し,Sahaの電離平衡式を解くことが求められる.

(2) \textbf{二線強度比法}  \\
 異なる状態にある同原子粒子からの放射強度比により温度が決定される.  
熱平衡を仮定して Saha の電離式から粒子数 \(N\) を決定し,式 (\ref{eq:line_intensity}) の放射強度比と温度を対応づける.

(3) \textbf{相対強度比法 (ボルツマンプロット法)}  \\
 同状態にある粒子のスペクトル強度比から温度を導出する.  
Saha の電離式を使用しないが,粒子間でボルツマン分布が成立するという仮定を設ける必要がある.  
二準位をそれぞれ 1,2として,式(\ref{eq:line_intensity})の放射強度を計算する.  
放射強度比と温度の関係は次式のようになるため,温度が決定される\cite{平岡和雄1996各種分光法によるアークプラズマの温度評価}.

\begin{equation}
T
=
\frac{E_2 - E_1}
{k \ln
\left(
\frac{\nu_2 A_2 g_2 I_1}
{\nu_1 A_1 g_1 I_2}
\right)}
\label{eq:boltzmann_ratio}
\end{equation}


\subsection{空間粒子組成計算}
局所熱平衡状態 (\textit{Local Thermodynamic Equilibrium, LTE}) が成り立つ場合には,
温度と圧力が決まれば,熱力学第二法則に従って粒子組成状態は系のエントロピーが最大となるように決定される.  
具体的に熱平衡状態における粒子組成を計算するには,次の2つの熱統計力学的手法がよく用いられる.

(1) 系の Gibbs 自由エネルギー \(G\) を求め,これを最小化させるように粒子組成を求める.  

(2) 状態方程式のもと,解離平衡についての \textit{Guldberg-Waage の式} と電離平衡についての \textit{Saha の式} を連立させて各粒子密度を解く.

これらをいずれも状態方程式が成り立つという制約条件の下で解くことにより粒子組成が得られる.上記2つの方法は等価な関係である \cite{1520853832517707008}.

Gibbs の自由エネルギー \(G\) は次式で与えられる.
\begin{equation}
G = \sum_{j=1}^{L} y_j \, 
\Biggl[
  \mu_j^0 
  + R_\mathrm{un} T \, 
  \ln \Biggl(
    \frac{y_j}{\sum_{p=1}^{L} y_p} 
    + \frac{P}{P_0}
  \Biggr)
\Biggr]
\end{equation}

ここで,  
\(y_j\):粒子 \(j\) のモル数 [mol],  
\(R = 8.31~\mathrm{J/mol/K}\):普遍気体定数,  
\(T\):温度 [K],  
\(L\):考慮する粒子種数,  
\(P\):圧力 [Pa],  
\(P_0\):標準圧力 (101325 Pa),  
\(\mu_j^0\):粒子 \(j\) の化学ポテンシャル [J/mol] であり,次式で与えられる.

\begin{equation}
\mu_j^0
= - R_\mathrm{un} T \,
\ln \! \Biggl[
  \left( \frac{2 \pi m_j k T}{h^2} \right)^{3/2} 
  Z_j^\mathrm{int}(T) \frac{k T}{P_0}
\Biggr] 
+ \Delta H_{fj}
\end{equation}

分子 \( \mathrm{AB} \) と粒子 \( \mathrm{A}, \mathrm{B} \) との解離平衡に関する Guldberg-Waage の式は次式となる.

\begin{equation}
\frac{n_{\mathrm{A}} \, n_{\mathrm{B}}}{n_{\mathrm{AB}}} 
=
\frac{(2 \pi m \, M_{\mathrm{AB}} k T)^{3/2}}{h^3} 
\frac{Z_{\mathrm{A}} Z_{\mathrm{B}}}{Z_{\mathrm{AB}}} 
\exp\Biggl( - \frac{E_{\mathrm{AB}}^\mathrm{dis}}{kT} \Biggr)
\end{equation}

ここで,  
\(n_{\mathrm{B}}, n_{\mathrm{AB}}\) はそれぞれ粒子 \( \mathrm{B}, \mathrm{AB} \) の数密度,  
\(Z_{\mathrm{B}}, Z_{\mathrm{AB}}\) はそれぞれ粒子 \( \mathrm{B}, \mathrm{AB} \) の分配関数(状態和),  
\(E_{\mathrm{AB}}^\mathrm{dis}\) は粒子 \( \mathrm{AB} \) の解離エネルギー,  
\(M_{\mathrm{AB}}\) は換算質量 \(\displaystyle M_{\mathrm{AB}} = \frac{m_{\mathrm{A}} m_{\mathrm{B}}}{m_{\mathrm{A}} + m_{\mathrm{B}}}\) である.

次に,粒子 \( \mathrm{A} \) の電離平衡に関する Saha の式は次式となる.

\begin{equation}
\frac{n_{\mathrm{A^+}} \, n_e}{n_{\mathrm{A}}} 
=
\frac{(2 \pi m_e k T)^{3/2}}{h^3} 
\frac{2 Z_{\mathrm{A^+}}}{Z_{\mathrm{A}}} 
\exp\Biggl( - \frac{E_{\mathrm{A}}^\mathrm{ion}}{kT} \Biggr)
\end{equation}

ここで,  
\(n_{\mathrm{A}}, n_{\mathrm{A^+}}, n_e\) はそれぞれ粒子 \( \mathrm{A}, \mathrm{A^+} \) および電子 \( e \) の数密度,  
\(Z_{\mathrm{A}}, Z_{\mathrm{A^+}}\) はそれぞれ粒子 \( \mathrm{A}, \mathrm{A^+} \) の分配関数,  
\(E_{\mathrm{A}}^\mathrm{ion}\) は粒子 \( \mathrm{A} \) の電離エネルギーである.


気体の状態方程式,プラズマとしての電気的中性条件,
元素比の式を考慮し,Newton-Raphson 法を基本とした連立方程式を解くことで粒子組成が得られる\cite{作田忠裕1978銅蒸気混入による高温空気中の電子密度の増大}.
\subsection{導電率算出}
Chapman-Enskog 法の第一近似を用いることにより,導電率は次式で表される.

\begin{equation}
\sigma =
\frac{
  \dfrac{3 e^2}{16 k T} \, n_e 
  \left( \dfrac{2 \pi k T}{m_e} \right)^{1/2}
}{
  \sum\nolimits_{j=1}^{n}{}' n_j \, \pi \, \bar{\Omega}_{ej}^{(1,1)}
}
\end{equation}



ここで,  
\(e\):電気素量,  
\(k\):ボルツマン定数,  
\(T\):絶対温度,  
\(m_e\):電子の質量,  
\(n_e\):電子密度,  
\(n_j\):粒子 \(j\) の数密度,  
\(\bar{\Omega}_{ej}^{(1,1)}\):電子と粒子 \(j\) との間の衝突断面積(拡散断面積),  
ここで,\(\sum\nolimits_{j=1}^{n}{}'\) は,和をとる際に
\(n_e \, \pi \, \bar{\Omega}_{ej}^{(1,1)}\) を除くことを意味する.


この式は,気中アークの銅蒸気を含む高温空気の輸送特性の解析や,アーク溶接の金属蒸気の挙動解析,酸素燃焼火炎中の電子伝導率の測定に用いられている.

\clearpage

% --- 第3章 ---
\chapter{実験手法}
\section{実験装置}
\subsection{気中開閉器}
図\ref{fig:気中開閉器}に,本試験で使用した気中開閉器の内部構造を示す.
本開閉器の可動電極は Cu 製(先端部は Cu--W 製),
アーキング電極は Fe 製である.
可動電極と Cu 製固定電極が接触した状態から遮断動作が開始されると,
可動電極は下方に回動して接点が引き離され,
電極間にアークが発生する.
発生したアークは,可動電極とアーキング電極間へ移行し,
遮断完了まで維持される.

消弧室内の消弧グリッドは,
板状の鉄製消弧板と
POM(ポリオキシメチレン)製絶縁板とを
一定間隔で交互に配置した積層構造を有する.
アークは消弧板に接触しながら分割され,
その際に発生した熱エネルギーにより,
周囲気体および消弧材料が高温ガス化する.
生成された高温ガスは,
図中橙色で示した領域に向かって噴出・滞留する.
\begin{figure}[H]
  \centering
  \includegraphics[width=\linewidth]{気中開閉器.png}
  \caption{気中開閉器の内部構造}
  \label{fig:気中開閉器}
\end{figure}
\FloatBarrier

\begin{figure}[H]
  \centering
  \includegraphics[width=0.6\linewidth]{試験回路.png}
  \caption{試験回路構成}
  \label{fig:試験回路}
\end{figure}

\subsection{試験回路}
図\ref{fig:試験回路}に,本実験における回路構成を示す.
本試験は,発電機容量 200\,MVA の短絡発電機を用い,
JIS C 4605 に準拠した試験手法により実施した.
ただし,試験用開閉器については 1 相のみ通電し,開閉試験を行った.

電流測定は,開閉器の電源側線路に電流センサーを設置して行った.
一方,電圧測定には高電圧プローブを用い,
開閉器の電源側および負荷側それぞれにおいて,
試験相と接地間の電位差を測定した.
これらの測定結果をもとに,
電源側および負荷側の電位差の差分を算出することで,
同相(試験相)における電極間電圧を求めた.


\subsection{メモリハイコーダー}
動作中の電流および電圧の測定には,MR6000-01 を使用した.
その主な仕様を表\ref{tab:メモハイ表}に示す.また,外観を図\ref{fig:メモハイ画像}に示す.
電流測定は,開閉器の電源側電線部に直流電流センサーを設置して行った.
電圧測定には高電圧プローブを用い,
開閉器の電源側および負荷側それぞれにおいて,
試験相と接地間の電位差を測定した.
得られた電源側および負荷側の電位差から差分を算出することで,
同相(試験相)における電極間電圧を観測した.

\begin{table}[H]
  \centering
  \caption{メモリハイコーダの規格}
  \label{tab:メモハイ表}
  \includegraphics[width=0.8\linewidth]{メモハイ表.png}
\end{table}

\begin{figure}[H]
  \centering
  \includegraphics[width=0.5\linewidth]{メモハイ画像.png}
  \caption{メモハイ画像}
  \label{fig:メモハイ画像}
\end{figure}

\begin{table}[H]
  \centering
  \caption{高速度カメラの仕様}
  \label{tab:高速度カメラ表}
  \includegraphics[width=0.8\linewidth]{高速度カメラ表.png}
\end{table}

\begin{figure}[H]
  \centering
  \includegraphics[width=0.5\linewidth]{高速度カメラ.png}
  \caption{高速度カメラの外観}
  \label{fig:高速度カメラ}
\end{figure}

\subsection{高速度カメラ}
2 次元分光画像解析を行うため,高速度カメラを用いて撮影を行った.
表\ref{tab:高速度カメラ表}にその主な仕様を,
図\ref{fig:高速度カメラ}に高速度カメラの外観を示す.



\subsection{パルスジェネレーター}
電流・電圧波形と高速度カメラ v2512,ならびに簡易分光器の
トリガ時間を同期させるため,パルスジェネレータを使用した.
本器の主な仕様を表\ref{tab:パルジェネ表}に示す.

メモリハイコーダからの立下り信号を閾値 $1\,\mathrm{V}$ で検出し,
パルス幅 $3\,\mu\mathrm{s}$,振幅 $3.3\,\mathrm{V}$ の
TTL 信号を出力することで,
各計測機器のトリガ時間を同期させた.

パルスジェネレータの外観を図\ref{fig:パルジェネ}に示す.




\subsection{狭帯域フィルタ}

測定に使用した 4 枚の狭帯域フィルタの仕様を
表\ref{tab:フィルタ}に示す.
いずれのフィルタも,半値全幅(FWHM)および
光学濃度(OD)が OD > 4.0 であり,
TECH SPEC 社製,直径 50\,mm のハードコートフィルタである.
この場合,中心波長以外の成分の透過率は 0.01\% 以下となる.

\begin{table}[H]
  \centering
  \caption{パルスジェネレーターの仕様}
  \label{tab:パルジェネ表}
  \includegraphics[width=0.8\linewidth]{パルジェネ表.png}
\end{table}

\begin{figure}[H]
  \centering
  \includegraphics[width=0.5\linewidth]{パルジェネ.png}
  \caption{パルスジェネレーターの外観}
  \label{fig:パルジェネ}
\end{figure}

\begin{table}[H]
  \centering
  \caption{狭帯域フィルタの商品コード}
  \label{tab:フィルタ}
  \includegraphics[width=0.8\linewidth]{フィルタ.png}
\end{table}
\clearpage
\section{分光画像センサー}
\subsection{概略}
局所熱平衡(Local Thermodynamic Equilibrium:LTE)仮定のもと,
プラズマからの複数の線スペクトル強度を用いて,
プラズマ温度,金属蒸気の混入率,および導電率の
時空間的変化を測定する手法は,これまでに数多く報告されている.
近年では,中心波長の異なる複数の狭帯域フィルタと
ハイスピードカメラを組み合わせることで,
線スペクトル強度の二次元分布を
高い時間・空間分解能で取得することが可能となっている\cite{inada2014temperature,kumada2019two}.
本研究では,これらの測定技術のうち,
アークプラズマに対して適用された手法を参考とし,
同様のアプローチを採用した.

フィルタ選定のため,
気中開閉器内に発生させたアーク放電の
発光分光スペクトルを図\ref{fig:発光スペクトル}に示す.
スペクトル中に Fe の線スペクトルが多く観測されるのは,
消弧板およびアーキング電極が Fe 製であることに起因している.
本研究では,使用する線スペクトルの選定にあたり,
(i) 強度,
(ii) スペクトル同士の近接度,
(iii) 上準位エネルギー差,
(iv) アーク温度に対する強度の変化率,
の 4 点を考慮した.

図\ref{fig:フローチャート}に示すように,
アーク温度の算出には 2 線強度比法\cite{boulos2013thermal}を用い,
Fe\,I スペクトルのうち,
$640\,\mathrm{nm}$ 近傍の線スペクトル群と
$650\,\mathrm{nm}$ 近傍の線スペクトル群を採用した.
また,鉄蒸気混入率および導電率の算定のため,
$777\,\mathrm{nm}$ 近傍の O\,I スペクトル群を採用した.
さらに,これらの Fe\,I および O\,I の線スペクトルには
連続スペクトルが重畳しているため,
$700\,\mathrm{nm}$ の連続スペクトルをバックグラウンドとして採用し,
その強度を差し引くことで,
純粋な線スペクトル強度を抽出した.

以上の検討を踏まえ,
中心波長が
$640\,\mathrm{nm}$,
$650\,\mathrm{nm}$,
$700\,\mathrm{nm}$,
$780\,\mathrm{nm}$,
半値全幅 $10\,\mathrm{nm}$ の干渉フィルタを
図\ref{fig:分光センサ}に示すように配置し,
各波長帯域に対応する分光画像の取得が可能な
分光画像センサーを構成した.
撮影には,Vision Research 社製
ハイスピードカメラ v2512 を用いた.
露光時間は実測された発光強度を確認しながら
$1\text{--}3\,\mu\mathrm{s}$ の範囲で設定し
(主な設定値は $1\,\mu\mathrm{s}$ および $3\,\mu\mathrm{s}$),
フレーム間隔を
$13.3\,\mu\mathrm{s}$ として
画像取得を行った.


\begin{figure}[H]
  \centering
  \includegraphics[width=0.7\linewidth]{発光スペクトル.png}
  \caption{発光分光スペクトル}
  \label{fig:発光スペクトル}
\end{figure}

\begin{figure}[H]
  \centering
  \includegraphics[width=0.6\linewidth]{フローチャート.png}
  \caption{スペクトル解析のフローチャート}
  \label{fig:フローチャート}
\end{figure}

\begin{figure}[H]
  \centering
  \includegraphics[width=0.8\linewidth]{分光センサ.png}
  \caption{分光画像センサー}
  \label{fig:分光センサ}
\end{figure}




\subsection{アーク温度}

本研究では,2 線強度比法\cite{boulos2013thermal}を用いて
アーク温度を算定した.
ある原子軌道中の電子がエネルギー準位 $E_{\mathrm{u}}$ から
より低いエネルギー準位 $E_{\mathrm{d}}$ へ遷移する際に放射される
線スペクトルの放射係数 $\varepsilon$ は,
次式で与えられる.
\begin{equation}
\varepsilon(\lambda)
= \frac{1}{4\pi}\frac{hc}{\lambda}
A_{\mathrm{u}\mathrm{d}}\, n_{\mathrm{u}}
\label{eq:emissivity}
\end{equation}
ここで,$\lambda\,[\mathrm{m}]$ は波長,
$h\,[\mathrm{J\,s}]$ はプランク定数,
$c\,[\mathrm{m/s}]$ は真空中の光速,
$A_{\mathrm{u}\mathrm{d}}\,[\mathrm{s^{-1}}]$ は
アインシュタインの $A$ 係数,
$n_{\mathrm{u}}\,[\mathrm{m^{-3}}]$ は
励起準位 $\mathrm{u}$ にある粒子の空間数密度である.

$n_{\mathrm{u}}$ は,上準位エネルギー
$E_{\mathrm{u}}\,[\mathrm{J}]$ を用いると,
$\mathrm{LTE}$ 仮定のもとで次式のように表される.
\begin{equation}
n_{\mathrm{u}}
= \frac{g_{\mathrm{u}}}{Z_{\mathrm{A}}(T)}\, n_{\mathrm{A}}
\exp\!\left(-\frac{E_{\mathrm{u}}}{k_{\mathrm{B}} T}\right)
\label{eq:boltzmann}
\end{equation}
ここで,$g_{\mathrm{u}}$ は準位 $E_{\mathrm{u}}$ の縮退度,
$T\,[\mathrm{K}]$ はアーク温度,
$Z_{\mathrm{A}}(T)$ は発光種の内部分配関数,
$n_{\mathrm{A}}\,[\mathrm{m^{-3}}]$ は
発光種の総空間数密度,
$k_{\mathrm{B}}\,[\mathrm{J/K}]$ は
ボルツマン定数である.

式\eqref{eq:emissivity}に式\eqref{eq:boltzmann}を代入し,
異なる 2 波長 $\lambda_1$ および $\lambda_2$ における
放射係数 $\varepsilon(\lambda_1)$,
$\varepsilon(\lambda_2)$ を同時に測定することで,
アーク温度 $T$ は次式から求めることができる.
\begin{equation}
T
= \frac{E_2 - E_1}
{k_{\mathrm{B}}
\ln\!\left(
\frac{
g_2 A_2 \lambda_1 \varepsilon(\lambda_1)
}{
g_1 A_1 \lambda_2 \varepsilon(\lambda_2)
}
\right)}
\label{eq:two_line_ratio}
\end{equation}
ここで,下付き添字 $\mathrm{1}$,$\mathrm{2}$ はそれぞれ
$\lambda_1$,$\lambda_2$ に対応する物理量を表す.

図\ref{fig:放射係数比と温度}に,
式\eqref{eq:two_line_ratio}に基づき算出した,
$640\,\mathrm{nm}$ 近傍および
$650\,\mathrm{nm}$ 近傍に存在する
Fe\,I スペクトル群の放射係数比
$\varepsilon_{640}/\varepsilon_{650}$ と
アーク温度 $T$ の関係を示す.
ここで,放射係数は各波長帯域に含まれる
スペクトル群の放射係数の総和として定義した.

電流零点近傍において
アークおよび熱ガスが取りうる温度領域
$T < 12000\,\mathrm{K}$ では,
$T$ は $\varepsilon_{640}/\varepsilon_{650}$ に対して
ほぼ線形に変化する.
一方,$T > 13000\,\mathrm{K}$ の領域では,
$\varepsilon_{640}/\varepsilon_{650}$ に対して
$T$ が複数の値を取りうるため,
温度の一意的な決定が困難となる.

なお,$\varepsilon_{640}/\varepsilon_{650}$ の測定精度は
およそ $\pm 15\%$ であるため,
本手法によるアーク温度 $T$ の測定精度も
同程度であると評価される.





\subsection{鉄蒸気混入率}

大気圧($0.1\,\mathrm{MPa}$)下において局所熱平衡(LTE)が成り立つと仮定し,
特定のアーク温度 $T$ および鉄蒸気混入率 $X_{\mathrm{Fe}}$ に対して,
ギブスの自由エネルギー最小化に基づく粒子組成計算を行った
\cite{tanaka2008thermodynamic}.
図\ref{fig:粒子組成}に,
$X_{\mathrm{Fe}} = 10^{-5}$ における粒子組成を一例として示す.

このとき,得られた粒子組成を用いることで,
式\eqref{eq:emissivity}および式\eqref{eq:boltzmann}より,
放射係数比 $\varepsilon_{640}/\varepsilon_{777}$ を
事前に計算することができる.
$X_{\mathrm{Fe}}$ をパラメータとした場合の
$\varepsilon_{640}/\varepsilon_{777}$ の温度依存性を
図\ref{fig:組成温度特性}に示す.

前節においてアーク温度 $T$ は既に求められているため,
実測された放射係数比 $\varepsilon_{640}/\varepsilon_{777}$ を用いることで,
鉄蒸気混入率 $X_{\mathrm{Fe}}$ を同定することが可能である.
図\ref{fig:組成温度特性}に示すように,
各曲線は $X_{\mathrm{Fe}}$ を $1$ 桁ずつ変化させて設定しており,
それらは互いに隣接している.
この傾向は温度が低いほど顕著となるため,
$X_{\mathrm{Fe}}$ の算定精度は
アーク温度 $T$ の測定精度に大きく依存する.

代表的な条件として,
$T = 6000\,\mathrm{K}$,
$X_{\mathrm{Fe}} = 10^{-4}$ の場合を考えると,
温度測定誤差に起因して,
鉄蒸気混入率の算定精度は
おおよそ $\pm 1$ 桁程度となる.

\begin{figure}[H]
  \centering
  \includegraphics[width=0.7\linewidth]{放射係数比と温度.png}
  \caption{放射係数比 $\varepsilon_{640}/\varepsilon_{650}$ とプラズマ温度 $T$ の関係
}
  \label{fig:放射係数比と温度}
\end{figure}

\begin{figure}[H]
  \centering
  \includegraphics[width=0.7\linewidth]{粒子組成.png}
  \caption{鉄蒸気混入空気の熱平衡粒子組成
  ($1\,\mathrm{bar}$,$X_{\mathrm{Fe}} = 10^{-5}$)}
  \label{fig:粒子組成}
\end{figure}

\begin{figure}[H]
  \centering
  \includegraphics[width=0.8\linewidth]{組成温度特性.png}
  \caption{プラズマ温度 $T$ と放射係数比
  $\varepsilon_{640}/\varepsilon_{777}$ の関係}
  \label{fig:組成温度特性}
\end{figure}

\subsection{導電率} 

各種構成粒子間における衝突断面積および内部状態和の
データセット\cite{nist_asd}を用い,
Chapman--Enskog 法の第一近似により
導電率 $\sigma$ を算定した\cite{tanaka2008thermodynamic}.
$X_{\mathrm{Fe}}$ をパラメータとした場合の
導電率 $\sigma$ のアーク温度 $T$ 依存性を
図\ref{fig:導電率}に示す.

前節で求めたアーク温度 $T$ および
鉄蒸気混入率 $X_{\mathrm{Fe}}$ に基づき,
対応する導電率 $\sigma$ は一意に定まる.
なお,本研究で測定対象とする熱ガスには,
消弧室を構成する POM の
アブレーション蒸気が混入している可能性がある.
しかし,$\sigma$ の温度依存性は
POM 蒸気と空気でほぼ同一であることが知られているため\cite{nakano2021numerical},
POM 蒸気の混入が
導電率の算定値に与える影響は
無視できるほど小さいと考えられる.

図\ref{fig:導電率}より,
$10^{-5} < X_{\mathrm{Fe}} < 10^{-3}$ の範囲では,
$\sigma$ の測定精度は主に
アーク温度 $T$ に依存し,
おおよそ $\pm 40\%$ 程度である.
一方で,
$10^{-2} < X_{\mathrm{Fe}} < 1$ の場合,
特に $T < 10000\,\mathrm{K}$ の温度領域において,
$\sigma$ の値は
$X_{\mathrm{Fe}}$ に強く依存する.

代表的な条件として,
$T = 6000\,\mathrm{K}$ において
$\sigma = 10^{3}\,\mathrm{S/m}$ と算定された場合,
測定誤差に起因する
導電率の算定精度は
おおよそ $\pm 1$ 桁程度となる.

\begin{figure}[H]
  \centering
  \includegraphics[width=0.8\linewidth]{導電率.png}
  \caption{鉄蒸気混入空気の導電率 $\sigma$($p = 1\,\mathrm{bar}$)}
  \label{fig:導電率}
\end{figure}
\clearpage


% --- 第4章 ---
\chapter{実験結果と考察}
\section{再発弧の分類}
\label{sec:再発弧の分類}
本研究では,開極動作がランダムな電流位相タイミングで開始される,175回の負荷電流遮断試験を実施した.
試験は60 Hzで実施し,開極動作開始から最初の電流ゼロ点(以下,第一電流ゼロ点)までの時間は,0~約8.33 msの範囲に分布する.
本稿では,遮断動作における第一電流ゼロ点での遮断成否を重点的に検討した.
あわせて,図\ref{fig:ゼロ点時間定義}のように,開極から第一電流ゼロ点までの経過時間を本研究において「ゼロ点時間」と定義する.\\
電流零点後に生じる再発弧は,残留電流およびエネルギー注入に起因する熱的再発弧と,
過渡回復電圧の上昇に伴う絶縁領域の誘電的破壊による再発弧に大別される.
これに基づき,本研究では第一電流零点における遮断成否を,以下の3パターンに分類した.

\begin{itemize}
 \item 再発弧を伴わない遮断成功
 \item 熱的再発弧
 \item 誘電的破壊を伴う再発弧
\end{itemize}

なお,熱的再発弧および誘電的再発弧の判別は,
再発弧時のアーク経路の変化,過渡回復電圧の強度,
および電極間に形成される絶縁層の有無に基づいて行った.

表\ref{tab:zc_classification}に,実施した全試験結果について,
ゼロ点時間ごとの再発弧の分類を示す.
なお,零点時間が $0\sim3\,\mathrm{ms}$ の区間では,
電極間に形成されるアークの様相を十分に判別することができなかった.
ゼロ点時間が長くなるほど消弧室内におけるアーク冷却時間が長くなるため,
熱的再発弧は抑制される傾向にある.
一方で,冷却の進行に伴い絶縁回復が支配的となることから,
誘電的再発弧の発生割合は相対的に増加する傾向が見られる.
しかしながら,誘電的再発弧の発生率は,
熱的再発弧と比較して全体として低いことが分かる.
このため,本稿では主として熱的再発弧に着目し,
その発生要因について検討を行う.

\begin{figure}[H]
  \centering
  \includegraphics[width=0.7\linewidth]{ゼロ点時間定義.png}
  \caption{ゼロ点時間の定義 }
  \label{fig:ゼロ点時間定義}
\end{figure}

\begin{table}[H]
 \centering
 \caption{ゼロ点時間ごとの再発弧の分類}
 \label{tab:zc_classification}
 \begin{tabular}{c|r|r|r|r}
  \hline
  ゼロ点時間 [ms] & 試験回数 & 遮断成功回数 & 熱的再発弧 & 誘電的再発弧 \\
  \hline
  0--3     & 70 & 0  & --- & --- \\
  3--4     & 15 & 0  & 15  & 0   \\
  4--5     & 21 & 2  & 18  & 1   \\
  5--6     & 18 & 3  & 14  & 1   \\
  6--7     & 19 & 8  & 9   & 2   \\
  7--8     & 18 & 11 & 5   & 2   \\
  8--半波  & 9  & 6  & 3   & 0   \\
  第二ゼロ点以降遮断 & 5 & 0 & 4 & 1 \\
  \hline
 \end{tabular}
\end{table}

\clearpage
\section{遮断時間の分類}
\label{sec:遮断時間の分類}
\ref{sec:再発弧の分類}項において,ゼロ点時間を定義した.
ゼロ点時間が長い場合には,第一電流零点到達時における電極間距離が十分に確保されるため,
アークが消滅しやすく,遮断成功率が高くなる傾向にある.
一方,ゼロ点時間が短い場合には,
電極間距離が不十分な状態で第一電流零点を迎えるため,
第一電流零点で遮断失敗が生じやすい.

以上の観点から,本研究では,
ゼロ点時間が短いにもかかわらず遮断に成功したケースを「短時間遮断」と定義する.
第一電流零点での遮断に失敗した場合,
遮断機会は次の電流零点まで持ち越され,
その間アーク放電は持続する.
本稿では,遮断に成功するまでのアーク時間が
$15\,\mathrm{ms}$ 以上の場合を「長時間遮断」とし,
「短時間遮断」と「長時間遮断」の中間に該当するケースを
「通常遮断」と分類した.

また,詳細は後述するが,
「短時間遮断」に該当するケースのアーク挙動を調べた結果,
第一電流零点における可動電極上のアークスポット位置が,
第一電流零点での遮断成否に大きく影響していることが示唆された.
このため,表\ref{tab:遮断性能表}に,
アーク時間に対する遮断成否およびアークスポット位置を整理した.


\begin{table}[H]
  \centering
  \caption{ゼロ点時間に基づく遮断時間とスポット位置の分類   }
  \label{tab:遮断性能表}
  \includegraphics[width=0.8\linewidth]{遮断性能表.png}
\end{table}
\clearpage
\subsection{アークスポットとの相関}
\label{sec:アークスポットとの相関}
遮断時間の分類に基づき,
第一電流零点付近におけるアーク特性を比較したところ,
可動電極上に形成されるアークスポット位置に明確な差異が確認された.
そこで,第一電流零点直前における可動電極上のアークスポット位置と,
可動電極上端面との最短距離を
「アークスポット距離」と定義し,比較を行った.

その結果,特に短時間遮断では,
アークスポット距離が長くなる傾向が認められた.
一方,長時間遮断では,
第一電流零点においてアークスポットが
可動電極右上付近に集中する傾向が確認された.
さらに,表\ref{tab:遮断性能表}に示す
アークスポット位置分布からは,
$15$~$20\,\mathrm{mm}$ といった比較的遠方にスポットが形成された割合が,
短時間遮断において有意に高いことが示されている.
これらの結果は,
アークスポット距離が長くなるほど,
遮断成功に寄与する可能性を示唆している.

この傾向を詳細に検証するため,
図\ref{fig:ゼロ点時間とアーク時間の関係性}に示すように,
ほぼ同一のゼロ点時間を有する2つのケースを選定し,
各ケースにおけるアーク温度および導電率を比較した.
ケース(a)は,
第一電流零点におけるアークスポット距離が短く,
遮断に失敗した例であり,
ケース(b)は,
アークスポット距離が長く,
第一電流零点で遮断に成功した例である.

各ケースのゼロ点直前における電流および電位差を図\ref{fig:電流ゼロ点直前の電流と電位差}に,
波長 $640\,\mathrm{nm}$ における発光強度分布に,
同時刻の可動電極位置を重ねて示した結果を図\ref{fig:電流ゼロ点直前のアーク発光比較}に,
温度および導電率分布を図\ref{fig:電流ゼロ点直前のアーク温度・導電率比較}に示す.
図\ref{fig:電流ゼロ点直前の電流と電位差}では,
第一電流零点を基準時刻 $t=0\,\mathrm{ms}$ としており,
各ケースのゼロ点時間は,
それぞれ $6.45\,\mathrm{ms}$ および $6.34\,\mathrm{ms}$ である.

\begin{figure}[H]
  \centering
  \includegraphics[width=0.6\linewidth]{ゼロ点時間とアーク時間の関係性.png}
  \caption{ゼロ点時間とアーク時間の関係性 }
  \label{fig:ゼロ点時間とアーク時間の関係性}
\end{figure}


\begin{figure}[H]
	\centering
	\begin{subfigure}[b]{0.48\columnwidth}
		\centering
		\includegraphics[width=0.8\linewidth]{ケースaIV.png}
		\subcaption{ケースa  }
		\label{fig:ケースaIV}
	\end{subfigure}
	\hfill
	\begin{subfigure}[b]{0.48\columnwidth}
		\centering
		\includegraphics[width=0.8\linewidth]{ケースbIV.png}
		\subcaption{ケースb  }
		\label{fig:ケースbIV}
	\end{subfigure}
	\caption{電流ゼロ点直前の電流と電位差}
	\label{fig:電流ゼロ点直前の電流と電位差}
\end{figure}

図\ref{fig:電流ゼロ点直前のアーク発光比較}(a),(b)より,
アーキング電極―可動電極間のスポット距離は,
それぞれ $20.2\,\mathrm{mm}$ および $43.2\,\mathrm{mm}$ であり,
ケース(b)はケース(a)の約 $2.1$ 倍に相当する.
また,図\ref{fig:電流ゼロ点直前のアーク温度・導電率比較}(a1),(b1)に示すアーク温度分布において,
白丸で示したアーク先端温度を比較すると,
ケース(a)では約 $4000$~$9000\,\mathrm{K}$,
ケース(b)では約 $4000$~$5000\,\mathrm{K}$ となっており,
スポット距離が長いケース(b)の方が
顕著に低温であることが確認された.
同様に,
図\ref{fig:電流ゼロ点直前のアーク温度・導電率比較}(a2),(b2)に示す導電率分布においても,
ケース(a)が $10^{-1}$~$10^{3.5}\,\mathrm{S/m}$,
ケース(b)が $10^{-1}$~$10^{2}\,\mathrm{S/m}$ の範囲にある.

したがって,アークスポットの距離が長くなることでアークが伸長しアーク抵抗が上昇した結果,
導電率が低く抑えられ,電流ゼロ点直後に注入される過渡回復電圧起因のジュールエネルギーも小さくなり熱的遮断に成功したものと考えられる.
さらに,図\ref{fig:電流ゼロ点直前のアーク温度・導電率比較}に示すように,
スポット距離が長くなることで,スポットとアーキング電極間に温度が $4000\,\mathrm{K}$ 未満の絶縁層も形成されやすくなり,
誘電的遮断にも成功しやすくなったと考えられる.

また,本実験器における可動電極先端は
Cu--W(銅・タングステン合金)製であるため,
アークスポット形成位置により電極材料が異なる.
具体的に,ケース(a)では主として Cu--W 上に,
ケース(b)では Cu 上にスポットが形成されている.
タングステンは銅と比較して融点が高く,
融解・蒸発に伴う潜熱による冷却効果が小さいことから,
これらの材料特性の差異も,アーク温度低下に寄与した一因であると考えられる.

\begin{figure}[H]
	\centering
	\begin{subfigure}[b]{0.48\columnwidth}
		\centering
		\includegraphics[width=0.8\linewidth]{ケースaEmission.png}
		\subcaption{ケースa}
		\label{fig:ケースaEmission}
	\end{subfigure}
	\hfill
	\begin{subfigure}[b]{0.48\columnwidth}
		\centering
		\includegraphics[width=0.8\linewidth]{ケースbEmission.png}
		\subcaption{ケースb}
		\label{fig:ケースbEmission}
	\end{subfigure}
	\caption{ケースa,bにおける電流ゼロ点直前の電流と電位差}
	\label{fig:電流ゼロ点直前のアーク発光比較}
\end{figure}

\begin{figure}[H]
  \centering
  % --- 上段 ---
  \begin{subfigure}{0.45\linewidth}
    \centering
    \includegraphics[width=\linewidth]{ケースaTemp.png}
    \caption*{(a1) ケースaのアーク温度}
    \label{fig:ケースaTemp}
  \end{subfigure}
  \hfill
  \begin{subfigure}{0.45\linewidth}
    \centering
    \includegraphics[width=\linewidth]{ケースbTemp.png}
    \caption*{(b1) ケースbのアーク温度}
    \label{fig:ケースbTemp}
  \end{subfigure}

  \vspace{3mm}

  % --- 下段 ---
  \begin{subfigure}{0.45\linewidth}
    \centering
    \includegraphics[width=\linewidth]{ケースacond.png}
    \caption*{(a2) ケースaの導電率}
    \label{fig:ケースacond}
  \end{subfigure}
  \hfill
  \begin{subfigure}{0.45\linewidth}
    \centering
    \includegraphics[width=\linewidth]{ケースbcond.png}
    \caption*{(b2) ケースbの導電率}
    \label{fig:ケースbcond}
  \end{subfigure}

  \caption{a,bにおける電流ゼロ点直前のアーク温度・導電率比較}
  \label{fig:電流ゼロ点直前のアーク温度・導電率比較}
\end{figure}
\FloatBarrier


\subsection{ガス発光との分類}
表\ref{tab:遮断性能表}に示した遮断速度の分類に基づき,「短時間遮断」を除くケースの特性を比較した.
その結果,ゼロ点時間が $7\,\mathrm{ms}$ 以上の条件において,第一電流ゼロ点直前における消弧室右上部のガス発光量に明確な差異が認められた.
そこで本研究では,図\ref{fig:ガス測定範囲}に示す白枠領域内において,発光強度がしきい値 $I_{\mathrm{th}}$ を上回る画素数を「高発光領域」の指標として定義し,評価を行った.
ここで \(I_{\mathrm{th}}\) は,
同一露光条件下において未飽和であった場合には最大発光強度の 75\%とし,
飽和が顕著な場合には撮像素子の飽和値と定義した.
また,図\ref{fig:発光強度分類境界}に示す発光強度分布の一例のように,
閾値以上の画素数に基づき,「小」「中」「大」の三段階に分類した.
なお,分類には鉄の発光スペクトルのうち,$640\,\mathrm{nm}$ よりも撮影範囲が広範な$650\,\mathrm{nm}$ の発光画像を用いた.

表\ref{table:ガス発光表}に,ゼロ点時間ごとの発光強度の分類結果を示す.
表\ref{table:ガス発光表}より,ゼロ点時間が $7\,\mathrm{ms}$ 以上の条件下において,発光強度が「大」に分類された割合に着目すると,
第一電流ゼロ点での遮断成功ケースでは16回中11回(約70$\%$)であったのに対し,再発弧したケースでは16回中2回(約12$\%$)にとどまり,両者の間には明確な差異が確認された.
この結果は,消弧室右上部におけるガス発光が顕著であるほど,第一電流ゼロ点での遮断成功に寄与する可能性を示唆している.
\begin{figure}[H]
  \centering
  \includegraphics[width=0.5\linewidth]{ガス測定範囲.png}
  \caption{ガス測定範囲}
  \label{fig:ガス測定範囲}
\end{figure}

\begin{figure}[H]
  \centering
  \includegraphics[width=0.6\linewidth]{発光強度分類境界.png}
  \caption{発光強度の分類}
  \label{fig:発光強度分類境界}
\end{figure}

\begin{table}[H]
  \centering

  \caption{ゼロ点時間ごとの発光の分類}
  \label{table:ガス発光表}
  \includegraphics[width=\linewidth]{ガス発光表.png}
\end{table}

この傾向を詳細に検討するため,\ref{sec:アークスポットとの相関}節と同様に,
ほぼ同一のゼロ点時間を有しながら遮断成否が分かれた2つの代表的なケースを選定し,これら両ケースにおける発光挙動,アーク温度分布,および導電率の比較を行った.

1つ目は,(c) 消弧室右上部の発光強度が弱く,
第一電流ゼロ点で遮断に失敗したケースであり,
2つ目は,
(d) 消弧室右上部の発光強度が強く,
第一電流ゼロ点で遮断に成功したケースである.

各ケースにおけるゼロ点直前の電流および電位差を
図\ref{fig:電流ゼロ点直前の電流と電位差2}に示す.
また,
波長650\,nmにおける発光強度分布に同時刻の可動電極位置を模式的に重ねて示した結果を
図\ref{fig:電流ゼロ点直前のアーク発光比較2}に,
温度および導電率分布を
図\ref{fig:電流ゼロ点直前のアーク温度・導電率比較2}に示す.

図\ref{fig:電流ゼロ点直前の電流と電位差2}では,
第一電流ゼロ点を基準時刻 $t = 0\,\mathrm{ms}$ としており,
各ケースのゼロ点時間は,
それぞれ $8.27\,\mathrm{ms}$ および $8.32\,\mathrm{ms}$ である.

図\ref{fig:電流ゼロ点直前のアーク発光比較2}(c)および(d)より,黄色枠で示した消弧室右上部の発光強度を比較した.
遮断に至らなかったケース(c)では,輝度値が約 $30,000$~$40,000$ の領域が分布するにとどまっている.
一方,遮断に成功したケース(d)では約 $50,000$ に達するより強い発光が確認され,前述の統計的傾向とも一致する結果となった.

同領域において図\ref{fig:電流ゼロ点直前のアーク温度・導電率比較2}(c1),(d1)に示す温度分布を比較すると,
ケース(d)はケース(c)に比べて
$8000\,\mathrm{K}$ 前後の高温領域が広範囲に分布しており,全体的に高温である.
具体的には,$8000\,\mathrm{K}$ 以上の領域の画素数は,
ケース(c)の296\,pxに対し,
ケース(d)で527\,pxであり,約1.8倍の高温領域を有している.
さらに,図\ref{fig:電流ゼロ点直前のアーク温度・導電率比較2}(c2)および(d2)に示す導電率分布においても,
ケース(d)はケース(c)に比べて$10^{3}$~$10^{4}\,\mathrm{S/m}$ の領域が広範囲に分布している.
当該領域の画素数は,ケース(c)の326\,pxに対し,ケース(d)では592\,pxであり,温度分布と同様に約1.8倍の分布面積を有している.
また,白枠で示した消弧室内のアーク経路を比較すると,右上枠内のガス発光が弱いケース(c)では,消弧室全体にわたってアーク経路が明瞭に確認できる.
これに対し,ガス発光の強いケース(d)では,特に可動電極近傍において発光が確認されず,温度および導電率が算出不可能なほど低温・低導電率領域となっている.
消弧室右上部へ噴出したガスが,高温・高導電率で強い発光を伴っていることは,アークの熱エネルギーが消弧室から効率的に排出されたことを意味する.
すなわち,右上部への熱エネルギー排出による主アーク経路上の温度および導電率の低下が,再発弧を抑制し遮断成功に至った要因であると考えられる.

\begin{figure}[H]
	\centering
	\begin{subfigure}[c]{0.48\columnwidth}
		\centering  
		\includegraphics[width=0.8\linewidth]{ケースcIV.png}
		\caption*{(c) ケースc}
		\label{fig:ケースcIV}
	\end{subfigure}
	\hfill
	\begin{subfigure}[d]{0.48\columnwidth}
		\centering
		\includegraphics[width=0.8\linewidth]{ケースdIV.png}
		\caption*{(d) ケースd}
		\label{fig:ケースdIV}
	\end{subfigure}
	\caption{c,dの電流ゼロ点直前の電流と電位差}
	\label{fig:電流ゼロ点直前の電流と電位差2}
\end{figure}

\begin{figure}[H]
	\centering
	\begin{subfigure}[c]{0.48\columnwidth}
		\centering
		\includegraphics[width=\linewidth]{ケースcEmission.png}
		\caption*{(c) ケースc}
		\label{fig:ケースcEmission}
	\end{subfigure}
	\hfill
	\begin{subfigure}[d]{0.48\columnwidth}
		\centering
		\includegraphics[width=\linewidth]{ケースdEmission.png}
		\caption*{(d) ケースd}
		\label{fig:ケースdEmission}
	\end{subfigure}
	\caption{c,dにおける電流ゼロ点直前のアーク発光比較}
	\label{fig:電流ゼロ点直前のアーク発光比較2}
\end{figure}

\begin{figure}[H]
  \centering
  % --- 上段 ---
  \begin{subfigure}{0.48\linewidth}
    \centering
    \includegraphics[width=\linewidth]{ケースcTemp.png}
    \caption*{(c1) ケースcのアーク温度}
    \label{fig:ケースcTemp}
  \end{subfigure}
  \hfill
  \begin{subfigure}{0.48\linewidth}
    \centering
    \includegraphics[width=\linewidth]{ケースdTemp.png}
    \caption*{(d1) ケースdのアーク温度}
    \label{fig:ケースdTemp}
  \end{subfigure}

  \vspace{3mm}

  % --- 下段 ---
  \begin{subfigure}{0.48\linewidth}
    \centering
    \includegraphics[width=\linewidth]{ケースccond.png}
    \caption*{(c2) ケースaの導電率}
    \label{fig:ケースccond}
  \end{subfigure}
  \hfill
  \begin{subfigure}{0.48\linewidth}
    \centering
    \includegraphics[width=\linewidth]{ケースdcond.png}
    \caption*{(d2) ケースbの導電率}
    \label{fig:ケースdcond}
  \end{subfigure}

  \caption{c,dにおける電流ゼロ点直前のアーク温度・導電率比較}
  \label{fig:電流ゼロ点直前のアーク温度・導電率比較2}
\end{figure}

\clearpage
これらを踏まえ,ゼロ点時間が 7,ms 以上で,かつ温度測定が可能であった 10 ケースを対象として,
消弧室右上の黄色枠内のガス領域におけるガス温度および導電率の総和と,アーク経路におけるそれらの総和との比(ガス/アーク経路)を算出した.
なお実験番号は図\ref{fig:発光強度分類境界}と同一であり,No.1 および No.5 はゼロ点時間が \qty{7}{\milli\second} 以下であったため対象から除外した.
図\ref{fig:温度・導電率総和比}より,第一電流ゼロ点で遮断に失敗したケースでは,温度および導電率のいずれにおいても総和比が小さい傾向が認められる.
具体的には,温度比が 25 以下,導電率比が 30 以下の場合に,第一電流ゼロ点での遮断失敗が生じている.
一方,遮断成功時の総和比は,それぞれの閾値の少なくとも 2 倍以上となっており,両者の間に明確な差異が存在することが分かる.
ここで,No.9 以降のケースでは,前述の通り消弧室右上部におけるガス発光の増大に起因して総和比も増大したと考えられる.
一方,ガス発光が比較的弱いにもかかわらず遮断に成功した No.3 および No.6 では,図\ref{fig:No.3,No.6の発光画像}に示すように,
電流ゼロ点直前におけるアーク経路自体の発光が十分に弱く,相対的に総和比が増大し,遮断成功に至ったものと推察される.
以上より,ゼロ点時間が $7\,\mathrm{ms}$ 以上の遮断においては,消弧室右上部への高エネルギー熱ガスの効率的な排出,
あるいはそれ以外の要因によってアーク経路の発光が十分に抑制されることが,遮断成否を分ける要因と考えられる.

\begin{figure}[htbp]
	\centering
	\begin{subfigure}[c]{0.48\columnwidth}
		\centering
		\includegraphics[width=\linewidth]{温度比.png}
		\caption*{ガス温度総和/アーク温度総和(比)}
		\label{fig:温度比}
	\end{subfigure}
	\hfill
	\begin{subfigure}[d]{0.48\columnwidth}
		\centering
		\includegraphics[width=\linewidth]{導電率比.png}
		\caption*{ガス導電率総和/アーク導電率総和(比)}
		\label{fig:導電率比}
	\end{subfigure}
	\caption{各実験番号におけるガス‐アーク温度・導電率総和比}
	\label{fig:温度・導電率総和比}
\end{figure}

\begin{figure}[htbp]
	\centering
	\begin{subfigure}[c]{0.48\columnwidth}
		\centering
		\includegraphics[width=\linewidth]{No.3発光.png}
		\caption*{(1)No.3}
		\label{fig:No.3発光}
	\end{subfigure}
	\hfill
	\begin{subfigure}[d]{0.48\columnwidth}
		\centering
		\includegraphics[width=\linewidth]{No.6発光.png}
		\caption*{(2)No.6}
		\label{fig:No.6発光}
	\end{subfigure}
	\caption{No.3,No.6の発光画像}
	\label{fig:No.3,No.6の発光画像}
\end{figure}
\clearpage

% --- 第5章 ---a
\chapter{結言}
\begin{itemize}
 \item 気中開閉器における再発弧の発生要因を解明するため,第一電流ゼロ点に着目し,アークおよび熱ガスの形態,温度,導電率の時空間的変化を測定した.
  \vspace{\baselineskip}
 \item 可動電極上のアークスポット位置と遮断速度との間には強い関連が認められ,遮断時間の短いケースでは,アークスポット位置が可動電極上で極間から遠ざかる傾向が顕著であった.
  \vspace{\baselineskip}
 \item スポット距離が大きい場合,アークスポット近傍における導電率は,遮断が成功したケースで \(10^{-1}\)~\(10^{2}\,\mathrm{S/m}\) と,失敗ケースに比べて約1桁低い値を示した.
  \vspace{\baselineskip}
 \item 以上の結果より,電流ゼロ点直前における可動電極上のアークスポット位置は,熱的遮断の成否を左右する重要な要因であることが示唆された.
  \vspace{\baselineskip}
 \item 開極から第一電流ゼロ点までの時間が長い場合,消弧室右上部におけるガス発光強度が強いほど,遮断成功確率が高くなる傾向が確認された.
  \vspace{\baselineskip}
 \item 遮断成功時には,消弧室右上部における高温・高導電率のガス空間が遮断失敗時の約1.8倍に拡大しており,あわせてアーク経路の発光が抑制されることで,
 アーク経路に対する消弧室右上ガスの温度および導電率の総和比が高くなることが分かった.
  \vspace{\baselineskip}
 \item 以上より,消弧室からの高温熱ガスの排出,アーク経路における発光の抑制,あるいはその両者が生じることで,遮断が成立しやすくなると考えられる.
\end{itemize}

\clearpage

% --- 謝辞 ---
\chapter*{謝辞}
 本研究を進めるにあたり,終始懇切丁寧なご指導と多くの有益なご助言を賜りました,稲田優貴准教授,前山光明教授に心より感謝申し上げます.
研究の方向性から論文執筆に至るまで,多大なるご指導を頂き,本研究をまとめることができました.
株式会社戸上電機製作所の西津章朗氏,山口直哉氏, 溝上智大氏には,試験の準備および実施に加え,発表練習や資料作成に関する相談に対しても親身にご対応いただきましたことに,厚く御礼申し上げます.
また実験など多くの場面においてご協力とご助言を頂いた,松田悠氏をはじめとした研究室の皆様に深く感謝いたします.実験および解析において多くの議論を通じて知見を深めることができました.
さらに,本研究に関して有益なご意見を賜りました関係各位に厚く御礼申し上げます.最後に,学生生活を支えてくれた家族ならびに友人に感謝の意を表し,本論文の謝辞といたします.
\clearpage

% --- 参考文献 ---

\bibliographystyle{unsrtnat}
\bibliography{reference}

\end{document}




